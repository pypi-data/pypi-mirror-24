% -*- coding: utf-8 -*-
%
% doc/tex/install.tex
%-------------------------------------------------- part of the fastmat demos
%
% Author      : sempersn
% Introduced  : 
%------------------------------------------------------------------------------
%  
%  Copyright 2016 Sebastian Semper, Christoph Wagner
%      https://www.tu-ilmenau.de/ems/
%
%  Licensed under the Apache License, Version 2.0 (the "License");
%  you may not use this file except in compliance with the License.
%  You may obtain a copy of the License at
%
%      http://www.apache.org/licenses/LICENSE-2.0
%
%  Unless required by applicable law or agreed to in writing, software
%  distributed under the License is distributed on an "AS IS" BASIS,
%  WITHOUT WARRANTIES OR CONDITIONS OF ANY KIND, either express or implied.
%  See the License for the specific language governing permissions and
%  limitations under the License.
%
%------------------------------------------------------------------------------
\subsection{Linux}
On a Debian derivative it is enough to call
\begin{lstlisting}
$ apt-get install python python-setuptools cython python-numpy python-scipy
\end{lstlisting}
in order to get the needed dependencies. Then we need to build the Cython
modules via
\begin{lstlisting}
$ python setup.py build_ext --inplace
\end{lstlisting}
from the folder where the source of \texttt{fastMat} was extracted. The final
step is to copy everything to your Python path. If you want to install
\fm{} globally issue (may need root privileges)
\begin{lstlisting}
$ python setup.py install 
\end{lstlisting}
If you want to only install it for your specific user, issue
\begin{lstlisting}
$ python setup.py install --user
\end{lstlisting}

If you want to build the documentation you might need some packages from the
latex distribution of your choice.

\subsection{Mac}
\subsection{Windows}
Dependencies:
\begin{enumerate}
\item Anaconda
\item Intel Python Distribution 2.7/3.5?
\item GNU Make
\item \LaTeX (Documentation)
\end{enumerate}
