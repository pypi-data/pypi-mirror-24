\documentclass[10pt]{article}

\newcommand\latexipy{\LaTeX iPy}
\newcommand\latexify{\LaTeX ify}

\usepackage[colorlinks]{hyperref}
\usepackage[nameinlink]{cleveref}
\hypersetup{
  linkcolor={red!50!black},
  citecolor={blue!50!black},
  urlcolor={blue!80!black},
  pdftitle={\latexipy\ Example},
  pdfsubject={\latexipy\ documentation},
  pdfauthor={Jean Nassar},
}

\usepackage{minted}
\setminted[latex]{
  autogobble=true,
}
\setminted[python]{
  autogobble=true,
  python3=true,
}
\newminted[pycode]{python}{}
\newmintinline[pyline]{python}{}
\newmintedfile[pyfile]{python}{}

\usepackage{pgf}

\title{\latexipy\ Example}
\author{Jean Nassar}
\date{\today}

\begin{document}
\maketitle
\tableofcontents
\listoflistings
\listoffigures

\section{Online resources}
The Github repository is at \url{https://github.com/masasin/latexipy},
and the full example file is at \href{https://github.com/masasin/latexipy/blob/master/examples/examples.py}{\texttt{examples/examples.py}}.
Full documentation is available at \url{https://latexipy.readthedocs.io/}.

\section{Assumptions}
In order to generate plots with \latexipy, install the package and import it.

This document assumes that the following imports are made:

\begin{listing}[H]
  \begin{pycode*}{highlightlines=3}
    from functools import partial
    import matplotlib.pyplot as plt
    import latexipy as lp
  \end{pycode*}
  \caption[List of imports]{The imports used in this example.}
  \label{lst:imports}
\end{listing}

Also, it assumes that there is a function, \pyline{plot_sin_and_cos()} which uses a Matplotlib-based package to generate a plot without calling \pyline{plt.savefig()} or \pyline{plt.close()}.
\latexipy\ is known to work well with various libraries, including Matplotlib, Numpy, Pandas and Seaborn, among others.

Finally, \pyline{figure()} in the listings (without \pyline{lp.}) is the partially applied version of the \pyline{lp.figure()} function to save it in the right directory.
See \Cref{sec:partial} for more details.

\section{Plotting}
\subsection{Without \latexify}
If you don't \latexify, Matplotlib's defaults are used.
The typeface is sans-serif, and the font a bit larger.
The code in \Cref{lst:sincos_no_latex} generates \Cref{fig:sincos_no_latex}.

\begin{listing}[H]
  \begin{pycode*}{highlightlines={1}}
    with figure('sincos'):
        plot_sin_and_cos()
  \end{pycode*}
  \caption[Figure generation]{Generate figures with one extra line.}
  \label{lst:sincos_no_latex}
\end{listing}

\begin{figure}[H]
  \centering
  %% Creator: Matplotlib, PGF backend
%%
%% To include the figure in your LaTeX document, write
%%   \input{<filename>.pgf}
%%
%% Make sure the required packages are loaded in your preamble
%%   \usepackage{pgf}
%%
%% Figures using additional raster images can only be included by \input if
%% they are in the same directory as the main LaTeX file. For loading figures
%% from other directories you can use the `import` package
%%   \usepackage{import}
%% and then include the figures with
%%   \import{<path to file>}{<filename>.pgf}
%%
%% Matplotlib used the following preamble
%%   \usepackage{fontspec}
%%   \setmainfont{DejaVu Serif}
%%   \setsansfont{DejaVu Sans}
%%   \setmonofont{DejaVu Sans Mono}
%%
\begingroup%
\makeatletter%
\begin{pgfpicture}%
\pgfpathrectangle{\pgfpointorigin}{\pgfqpoint{4.296389in}{2.655314in}}%
\pgfusepath{use as bounding box, clip}%
\begin{pgfscope}%
\pgfsetbuttcap%
\pgfsetmiterjoin%
\definecolor{currentfill}{rgb}{1.000000,1.000000,1.000000}%
\pgfsetfillcolor{currentfill}%
\pgfsetlinewidth{0.000000pt}%
\definecolor{currentstroke}{rgb}{1.000000,1.000000,1.000000}%
\pgfsetstrokecolor{currentstroke}%
\pgfsetdash{}{0pt}%
\pgfpathmoveto{\pgfqpoint{0.000000in}{0.000000in}}%
\pgfpathlineto{\pgfqpoint{4.296389in}{0.000000in}}%
\pgfpathlineto{\pgfqpoint{4.296389in}{2.655314in}}%
\pgfpathlineto{\pgfqpoint{0.000000in}{2.655314in}}%
\pgfpathclose%
\pgfusepath{fill}%
\end{pgfscope}%
\begin{pgfscope}%
\pgfsetbuttcap%
\pgfsetmiterjoin%
\definecolor{currentfill}{rgb}{1.000000,1.000000,1.000000}%
\pgfsetfillcolor{currentfill}%
\pgfsetlinewidth{0.000000pt}%
\definecolor{currentstroke}{rgb}{0.000000,0.000000,0.000000}%
\pgfsetstrokecolor{currentstroke}%
\pgfsetstrokeopacity{0.000000}%
\pgfsetdash{}{0pt}%
\pgfpathmoveto{\pgfqpoint{0.629028in}{0.442778in}}%
\pgfpathlineto{\pgfqpoint{4.261389in}{0.442778in}}%
\pgfpathlineto{\pgfqpoint{4.261389in}{2.431981in}}%
\pgfpathlineto{\pgfqpoint{0.629028in}{2.431981in}}%
\pgfpathclose%
\pgfusepath{fill}%
\end{pgfscope}%
\begin{pgfscope}%
\pgfsetbuttcap%
\pgfsetroundjoin%
\definecolor{currentfill}{rgb}{0.000000,0.000000,0.000000}%
\pgfsetfillcolor{currentfill}%
\pgfsetlinewidth{0.803000pt}%
\definecolor{currentstroke}{rgb}{0.000000,0.000000,0.000000}%
\pgfsetstrokecolor{currentstroke}%
\pgfsetdash{}{0pt}%
\pgfsys@defobject{currentmarker}{\pgfqpoint{0.000000in}{-0.048611in}}{\pgfqpoint{0.000000in}{0.000000in}}{%
\pgfpathmoveto{\pgfqpoint{0.000000in}{0.000000in}}%
\pgfpathlineto{\pgfqpoint{0.000000in}{-0.048611in}}%
\pgfusepath{stroke,fill}%
}%
\begin{pgfscope}%
\pgfsys@transformshift{0.868550in}{0.442778in}%
\pgfsys@useobject{currentmarker}{}%
\end{pgfscope}%
\end{pgfscope}%
\begin{pgfscope}%
\pgftext[x=0.868550in,y=0.345556in,,top]{\sffamily\fontsize{10.000000}{12.000000}\selectfont −3}%
\end{pgfscope}%
\begin{pgfscope}%
\pgfsetbuttcap%
\pgfsetroundjoin%
\definecolor{currentfill}{rgb}{0.000000,0.000000,0.000000}%
\pgfsetfillcolor{currentfill}%
\pgfsetlinewidth{0.803000pt}%
\definecolor{currentstroke}{rgb}{0.000000,0.000000,0.000000}%
\pgfsetstrokecolor{currentstroke}%
\pgfsetdash{}{0pt}%
\pgfsys@defobject{currentmarker}{\pgfqpoint{0.000000in}{-0.048611in}}{\pgfqpoint{0.000000in}{0.000000in}}{%
\pgfpathmoveto{\pgfqpoint{0.000000in}{0.000000in}}%
\pgfpathlineto{\pgfqpoint{0.000000in}{-0.048611in}}%
\pgfusepath{stroke,fill}%
}%
\begin{pgfscope}%
\pgfsys@transformshift{1.394102in}{0.442778in}%
\pgfsys@useobject{currentmarker}{}%
\end{pgfscope}%
\end{pgfscope}%
\begin{pgfscope}%
\pgftext[x=1.394102in,y=0.345556in,,top]{\sffamily\fontsize{10.000000}{12.000000}\selectfont −2}%
\end{pgfscope}%
\begin{pgfscope}%
\pgfsetbuttcap%
\pgfsetroundjoin%
\definecolor{currentfill}{rgb}{0.000000,0.000000,0.000000}%
\pgfsetfillcolor{currentfill}%
\pgfsetlinewidth{0.803000pt}%
\definecolor{currentstroke}{rgb}{0.000000,0.000000,0.000000}%
\pgfsetstrokecolor{currentstroke}%
\pgfsetdash{}{0pt}%
\pgfsys@defobject{currentmarker}{\pgfqpoint{0.000000in}{-0.048611in}}{\pgfqpoint{0.000000in}{0.000000in}}{%
\pgfpathmoveto{\pgfqpoint{0.000000in}{0.000000in}}%
\pgfpathlineto{\pgfqpoint{0.000000in}{-0.048611in}}%
\pgfusepath{stroke,fill}%
}%
\begin{pgfscope}%
\pgfsys@transformshift{1.919655in}{0.442778in}%
\pgfsys@useobject{currentmarker}{}%
\end{pgfscope}%
\end{pgfscope}%
\begin{pgfscope}%
\pgftext[x=1.919655in,y=0.345556in,,top]{\sffamily\fontsize{10.000000}{12.000000}\selectfont −1}%
\end{pgfscope}%
\begin{pgfscope}%
\pgfsetbuttcap%
\pgfsetroundjoin%
\definecolor{currentfill}{rgb}{0.000000,0.000000,0.000000}%
\pgfsetfillcolor{currentfill}%
\pgfsetlinewidth{0.803000pt}%
\definecolor{currentstroke}{rgb}{0.000000,0.000000,0.000000}%
\pgfsetstrokecolor{currentstroke}%
\pgfsetdash{}{0pt}%
\pgfsys@defobject{currentmarker}{\pgfqpoint{0.000000in}{-0.048611in}}{\pgfqpoint{0.000000in}{0.000000in}}{%
\pgfpathmoveto{\pgfqpoint{0.000000in}{0.000000in}}%
\pgfpathlineto{\pgfqpoint{0.000000in}{-0.048611in}}%
\pgfusepath{stroke,fill}%
}%
\begin{pgfscope}%
\pgfsys@transformshift{2.445208in}{0.442778in}%
\pgfsys@useobject{currentmarker}{}%
\end{pgfscope}%
\end{pgfscope}%
\begin{pgfscope}%
\pgftext[x=2.445208in,y=0.345556in,,top]{\sffamily\fontsize{10.000000}{12.000000}\selectfont 0}%
\end{pgfscope}%
\begin{pgfscope}%
\pgfsetbuttcap%
\pgfsetroundjoin%
\definecolor{currentfill}{rgb}{0.000000,0.000000,0.000000}%
\pgfsetfillcolor{currentfill}%
\pgfsetlinewidth{0.803000pt}%
\definecolor{currentstroke}{rgb}{0.000000,0.000000,0.000000}%
\pgfsetstrokecolor{currentstroke}%
\pgfsetdash{}{0pt}%
\pgfsys@defobject{currentmarker}{\pgfqpoint{0.000000in}{-0.048611in}}{\pgfqpoint{0.000000in}{0.000000in}}{%
\pgfpathmoveto{\pgfqpoint{0.000000in}{0.000000in}}%
\pgfpathlineto{\pgfqpoint{0.000000in}{-0.048611in}}%
\pgfusepath{stroke,fill}%
}%
\begin{pgfscope}%
\pgfsys@transformshift{2.970761in}{0.442778in}%
\pgfsys@useobject{currentmarker}{}%
\end{pgfscope}%
\end{pgfscope}%
\begin{pgfscope}%
\pgftext[x=2.970761in,y=0.345556in,,top]{\sffamily\fontsize{10.000000}{12.000000}\selectfont 1}%
\end{pgfscope}%
\begin{pgfscope}%
\pgfsetbuttcap%
\pgfsetroundjoin%
\definecolor{currentfill}{rgb}{0.000000,0.000000,0.000000}%
\pgfsetfillcolor{currentfill}%
\pgfsetlinewidth{0.803000pt}%
\definecolor{currentstroke}{rgb}{0.000000,0.000000,0.000000}%
\pgfsetstrokecolor{currentstroke}%
\pgfsetdash{}{0pt}%
\pgfsys@defobject{currentmarker}{\pgfqpoint{0.000000in}{-0.048611in}}{\pgfqpoint{0.000000in}{0.000000in}}{%
\pgfpathmoveto{\pgfqpoint{0.000000in}{0.000000in}}%
\pgfpathlineto{\pgfqpoint{0.000000in}{-0.048611in}}%
\pgfusepath{stroke,fill}%
}%
\begin{pgfscope}%
\pgfsys@transformshift{3.496314in}{0.442778in}%
\pgfsys@useobject{currentmarker}{}%
\end{pgfscope}%
\end{pgfscope}%
\begin{pgfscope}%
\pgftext[x=3.496314in,y=0.345556in,,top]{\sffamily\fontsize{10.000000}{12.000000}\selectfont 2}%
\end{pgfscope}%
\begin{pgfscope}%
\pgfsetbuttcap%
\pgfsetroundjoin%
\definecolor{currentfill}{rgb}{0.000000,0.000000,0.000000}%
\pgfsetfillcolor{currentfill}%
\pgfsetlinewidth{0.803000pt}%
\definecolor{currentstroke}{rgb}{0.000000,0.000000,0.000000}%
\pgfsetstrokecolor{currentstroke}%
\pgfsetdash{}{0pt}%
\pgfsys@defobject{currentmarker}{\pgfqpoint{0.000000in}{-0.048611in}}{\pgfqpoint{0.000000in}{0.000000in}}{%
\pgfpathmoveto{\pgfqpoint{0.000000in}{0.000000in}}%
\pgfpathlineto{\pgfqpoint{0.000000in}{-0.048611in}}%
\pgfusepath{stroke,fill}%
}%
\begin{pgfscope}%
\pgfsys@transformshift{4.021867in}{0.442778in}%
\pgfsys@useobject{currentmarker}{}%
\end{pgfscope}%
\end{pgfscope}%
\begin{pgfscope}%
\pgftext[x=4.021867in,y=0.345556in,,top]{\sffamily\fontsize{10.000000}{12.000000}\selectfont 3}%
\end{pgfscope}%
\begin{pgfscope}%
\pgftext[x=2.445208in,y=0.155587in,,top]{\sffamily\fontsize{10.000000}{12.000000}\selectfont \(\displaystyle \theta\)}%
\end{pgfscope}%
\begin{pgfscope}%
\pgfsetbuttcap%
\pgfsetroundjoin%
\definecolor{currentfill}{rgb}{0.000000,0.000000,0.000000}%
\pgfsetfillcolor{currentfill}%
\pgfsetlinewidth{0.803000pt}%
\definecolor{currentstroke}{rgb}{0.000000,0.000000,0.000000}%
\pgfsetstrokecolor{currentstroke}%
\pgfsetdash{}{0pt}%
\pgfsys@defobject{currentmarker}{\pgfqpoint{-0.048611in}{0.000000in}}{\pgfqpoint{0.000000in}{0.000000in}}{%
\pgfpathmoveto{\pgfqpoint{0.000000in}{0.000000in}}%
\pgfpathlineto{\pgfqpoint{-0.048611in}{0.000000in}}%
\pgfusepath{stroke,fill}%
}%
\begin{pgfscope}%
\pgfsys@transformshift{0.629028in}{0.533196in}%
\pgfsys@useobject{currentmarker}{}%
\end{pgfscope}%
\end{pgfscope}%
\begin{pgfscope}%
\pgftext[x=0.194552in,y=0.480435in,left,base]{\sffamily\fontsize{10.000000}{12.000000}\selectfont −1.0}%
\end{pgfscope}%
\begin{pgfscope}%
\pgfsetbuttcap%
\pgfsetroundjoin%
\definecolor{currentfill}{rgb}{0.000000,0.000000,0.000000}%
\pgfsetfillcolor{currentfill}%
\pgfsetlinewidth{0.803000pt}%
\definecolor{currentstroke}{rgb}{0.000000,0.000000,0.000000}%
\pgfsetstrokecolor{currentstroke}%
\pgfsetdash{}{0pt}%
\pgfsys@defobject{currentmarker}{\pgfqpoint{-0.048611in}{0.000000in}}{\pgfqpoint{0.000000in}{0.000000in}}{%
\pgfpathmoveto{\pgfqpoint{0.000000in}{0.000000in}}%
\pgfpathlineto{\pgfqpoint{-0.048611in}{0.000000in}}%
\pgfusepath{stroke,fill}%
}%
\begin{pgfscope}%
\pgfsys@transformshift{0.629028in}{0.985404in}%
\pgfsys@useobject{currentmarker}{}%
\end{pgfscope}%
\end{pgfscope}%
\begin{pgfscope}%
\pgftext[x=0.194552in,y=0.932642in,left,base]{\sffamily\fontsize{10.000000}{12.000000}\selectfont −0.5}%
\end{pgfscope}%
\begin{pgfscope}%
\pgfsetbuttcap%
\pgfsetroundjoin%
\definecolor{currentfill}{rgb}{0.000000,0.000000,0.000000}%
\pgfsetfillcolor{currentfill}%
\pgfsetlinewidth{0.803000pt}%
\definecolor{currentstroke}{rgb}{0.000000,0.000000,0.000000}%
\pgfsetstrokecolor{currentstroke}%
\pgfsetdash{}{0pt}%
\pgfsys@defobject{currentmarker}{\pgfqpoint{-0.048611in}{0.000000in}}{\pgfqpoint{0.000000in}{0.000000in}}{%
\pgfpathmoveto{\pgfqpoint{0.000000in}{0.000000in}}%
\pgfpathlineto{\pgfqpoint{-0.048611in}{0.000000in}}%
\pgfusepath{stroke,fill}%
}%
\begin{pgfscope}%
\pgfsys@transformshift{0.629028in}{1.437612in}%
\pgfsys@useobject{currentmarker}{}%
\end{pgfscope}%
\end{pgfscope}%
\begin{pgfscope}%
\pgftext[x=0.310926in,y=1.384850in,left,base]{\sffamily\fontsize{10.000000}{12.000000}\selectfont 0.0}%
\end{pgfscope}%
\begin{pgfscope}%
\pgfsetbuttcap%
\pgfsetroundjoin%
\definecolor{currentfill}{rgb}{0.000000,0.000000,0.000000}%
\pgfsetfillcolor{currentfill}%
\pgfsetlinewidth{0.803000pt}%
\definecolor{currentstroke}{rgb}{0.000000,0.000000,0.000000}%
\pgfsetstrokecolor{currentstroke}%
\pgfsetdash{}{0pt}%
\pgfsys@defobject{currentmarker}{\pgfqpoint{-0.048611in}{0.000000in}}{\pgfqpoint{0.000000in}{0.000000in}}{%
\pgfpathmoveto{\pgfqpoint{0.000000in}{0.000000in}}%
\pgfpathlineto{\pgfqpoint{-0.048611in}{0.000000in}}%
\pgfusepath{stroke,fill}%
}%
\begin{pgfscope}%
\pgfsys@transformshift{0.629028in}{1.889819in}%
\pgfsys@useobject{currentmarker}{}%
\end{pgfscope}%
\end{pgfscope}%
\begin{pgfscope}%
\pgftext[x=0.310926in,y=1.837058in,left,base]{\sffamily\fontsize{10.000000}{12.000000}\selectfont 0.5}%
\end{pgfscope}%
\begin{pgfscope}%
\pgfsetbuttcap%
\pgfsetroundjoin%
\definecolor{currentfill}{rgb}{0.000000,0.000000,0.000000}%
\pgfsetfillcolor{currentfill}%
\pgfsetlinewidth{0.803000pt}%
\definecolor{currentstroke}{rgb}{0.000000,0.000000,0.000000}%
\pgfsetstrokecolor{currentstroke}%
\pgfsetdash{}{0pt}%
\pgfsys@defobject{currentmarker}{\pgfqpoint{-0.048611in}{0.000000in}}{\pgfqpoint{0.000000in}{0.000000in}}{%
\pgfpathmoveto{\pgfqpoint{0.000000in}{0.000000in}}%
\pgfpathlineto{\pgfqpoint{-0.048611in}{0.000000in}}%
\pgfusepath{stroke,fill}%
}%
\begin{pgfscope}%
\pgfsys@transformshift{0.629028in}{2.342027in}%
\pgfsys@useobject{currentmarker}{}%
\end{pgfscope}%
\end{pgfscope}%
\begin{pgfscope}%
\pgftext[x=0.310926in,y=2.289266in,left,base]{\sffamily\fontsize{10.000000}{12.000000}\selectfont 1.0}%
\end{pgfscope}%
\begin{pgfscope}%
\pgftext[x=0.138997in,y=1.437379in,,bottom,rotate=90.000000]{\sffamily\fontsize{10.000000}{12.000000}\selectfont Value}%
\end{pgfscope}%
\begin{pgfscope}%
\pgfpathrectangle{\pgfqpoint{0.629028in}{0.442778in}}{\pgfqpoint{3.632361in}{1.989203in}} %
\pgfusepath{clip}%
\pgfsetrectcap%
\pgfsetroundjoin%
\pgfsetlinewidth{1.505625pt}%
\definecolor{currentstroke}{rgb}{0.121569,0.466667,0.705882}%
\pgfsetstrokecolor{currentstroke}%
\pgfsetdash{}{0pt}%
\pgfpathmoveto{\pgfqpoint{0.794135in}{1.437612in}}%
\pgfpathlineto{\pgfqpoint{0.861526in}{1.321958in}}%
\pgfpathlineto{\pgfqpoint{0.928917in}{1.208202in}}%
\pgfpathlineto{\pgfqpoint{0.996307in}{1.098214in}}%
\pgfpathlineto{\pgfqpoint{1.063698in}{0.993799in}}%
\pgfpathlineto{\pgfqpoint{1.131089in}{0.896671in}}%
\pgfpathlineto{\pgfqpoint{1.198480in}{0.808426in}}%
\pgfpathlineto{\pgfqpoint{1.265870in}{0.730511in}}%
\pgfpathlineto{\pgfqpoint{1.333261in}{0.664207in}}%
\pgfpathlineto{\pgfqpoint{1.400652in}{0.610603in}}%
\pgfpathlineto{\pgfqpoint{1.468042in}{0.570578in}}%
\pgfpathlineto{\pgfqpoint{1.535433in}{0.544789in}}%
\pgfpathlineto{\pgfqpoint{1.602824in}{0.533661in}}%
\pgfpathlineto{\pgfqpoint{1.670215in}{0.537375in}}%
\pgfpathlineto{\pgfqpoint{1.737605in}{0.555872in}}%
\pgfpathlineto{\pgfqpoint{1.804996in}{0.588846in}}%
\pgfpathlineto{\pgfqpoint{1.872387in}{0.635757in}}%
\pgfpathlineto{\pgfqpoint{1.939778in}{0.695835in}}%
\pgfpathlineto{\pgfqpoint{2.007168in}{0.768093in}}%
\pgfpathlineto{\pgfqpoint{2.074559in}{0.851344in}}%
\pgfpathlineto{\pgfqpoint{2.141950in}{0.944221in}}%
\pgfpathlineto{\pgfqpoint{2.209341in}{1.045200in}}%
\pgfpathlineto{\pgfqpoint{2.276731in}{1.152623in}}%
\pgfpathlineto{\pgfqpoint{2.344122in}{1.264725in}}%
\pgfpathlineto{\pgfqpoint{2.411513in}{1.379666in}}%
\pgfpathlineto{\pgfqpoint{2.478904in}{1.495558in}}%
\pgfpathlineto{\pgfqpoint{2.546294in}{1.610498in}}%
\pgfpathlineto{\pgfqpoint{2.613685in}{1.722600in}}%
\pgfpathlineto{\pgfqpoint{2.681076in}{1.830023in}}%
\pgfpathlineto{\pgfqpoint{2.748466in}{1.931002in}}%
\pgfpathlineto{\pgfqpoint{2.815857in}{2.023879in}}%
\pgfpathlineto{\pgfqpoint{2.883248in}{2.107131in}}%
\pgfpathlineto{\pgfqpoint{2.950639in}{2.179388in}}%
\pgfpathlineto{\pgfqpoint{3.018029in}{2.239466in}}%
\pgfpathlineto{\pgfqpoint{3.085420in}{2.286377in}}%
\pgfpathlineto{\pgfqpoint{3.152811in}{2.319352in}}%
\pgfpathlineto{\pgfqpoint{3.220202in}{2.337848in}}%
\pgfpathlineto{\pgfqpoint{3.287592in}{2.341562in}}%
\pgfpathlineto{\pgfqpoint{3.354983in}{2.330434in}}%
\pgfpathlineto{\pgfqpoint{3.422374in}{2.304646in}}%
\pgfpathlineto{\pgfqpoint{3.489765in}{2.264621in}}%
\pgfpathlineto{\pgfqpoint{3.557155in}{2.211016in}}%
\pgfpathlineto{\pgfqpoint{3.624546in}{2.144712in}}%
\pgfpathlineto{\pgfqpoint{3.691937in}{2.066798in}}%
\pgfpathlineto{\pgfqpoint{3.759328in}{1.978552in}}%
\pgfpathlineto{\pgfqpoint{3.826718in}{1.881424in}}%
\pgfpathlineto{\pgfqpoint{3.894109in}{1.777009in}}%
\pgfpathlineto{\pgfqpoint{3.961500in}{1.667021in}}%
\pgfpathlineto{\pgfqpoint{4.028890in}{1.553266in}}%
\pgfpathlineto{\pgfqpoint{4.096281in}{1.437612in}}%
\pgfusepath{stroke}%
\end{pgfscope}%
\begin{pgfscope}%
\pgfpathrectangle{\pgfqpoint{0.629028in}{0.442778in}}{\pgfqpoint{3.632361in}{1.989203in}} %
\pgfusepath{clip}%
\pgfsetrectcap%
\pgfsetroundjoin%
\pgfsetlinewidth{1.505625pt}%
\definecolor{currentstroke}{rgb}{1.000000,0.498039,0.054902}%
\pgfsetstrokecolor{currentstroke}%
\pgfsetdash{}{0pt}%
\pgfpathmoveto{\pgfqpoint{0.794135in}{0.533196in}}%
\pgfpathlineto{\pgfqpoint{0.861526in}{0.540621in}}%
\pgfpathlineto{\pgfqpoint{0.928917in}{0.562775in}}%
\pgfpathlineto{\pgfqpoint{0.996307in}{0.599294in}}%
\pgfpathlineto{\pgfqpoint{1.063698in}{0.649577in}}%
\pgfpathlineto{\pgfqpoint{1.131089in}{0.712801in}}%
\pgfpathlineto{\pgfqpoint{1.198480in}{0.787925in}}%
\pgfpathlineto{\pgfqpoint{1.265870in}{0.873718in}}%
\pgfpathlineto{\pgfqpoint{1.333261in}{0.968769in}}%
\pgfpathlineto{\pgfqpoint{1.400652in}{1.071519in}}%
\pgfpathlineto{\pgfqpoint{1.468042in}{1.180280in}}%
\pgfpathlineto{\pgfqpoint{1.535433in}{1.293267in}}%
\pgfpathlineto{\pgfqpoint{1.602824in}{1.408624in}}%
\pgfpathlineto{\pgfqpoint{1.670215in}{1.524456in}}%
\pgfpathlineto{\pgfqpoint{1.737605in}{1.638863in}}%
\pgfpathlineto{\pgfqpoint{1.804996in}{1.749965in}}%
\pgfpathlineto{\pgfqpoint{1.872387in}{1.855938in}}%
\pgfpathlineto{\pgfqpoint{1.939778in}{1.955043in}}%
\pgfpathlineto{\pgfqpoint{2.007168in}{2.045651in}}%
\pgfpathlineto{\pgfqpoint{2.074559in}{2.126275in}}%
\pgfpathlineto{\pgfqpoint{2.141950in}{2.195592in}}%
\pgfpathlineto{\pgfqpoint{2.209341in}{2.252462in}}%
\pgfpathlineto{\pgfqpoint{2.276731in}{2.295952in}}%
\pgfpathlineto{\pgfqpoint{2.344122in}{2.325349in}}%
\pgfpathlineto{\pgfqpoint{2.411513in}{2.340169in}}%
\pgfpathlineto{\pgfqpoint{2.478904in}{2.340169in}}%
\pgfpathlineto{\pgfqpoint{2.546294in}{2.325349in}}%
\pgfpathlineto{\pgfqpoint{2.613685in}{2.295952in}}%
\pgfpathlineto{\pgfqpoint{2.681076in}{2.252462in}}%
\pgfpathlineto{\pgfqpoint{2.748466in}{2.195592in}}%
\pgfpathlineto{\pgfqpoint{2.815857in}{2.126275in}}%
\pgfpathlineto{\pgfqpoint{2.883248in}{2.045651in}}%
\pgfpathlineto{\pgfqpoint{2.950639in}{1.955043in}}%
\pgfpathlineto{\pgfqpoint{3.018029in}{1.855938in}}%
\pgfpathlineto{\pgfqpoint{3.085420in}{1.749965in}}%
\pgfpathlineto{\pgfqpoint{3.152811in}{1.638863in}}%
\pgfpathlineto{\pgfqpoint{3.220202in}{1.524456in}}%
\pgfpathlineto{\pgfqpoint{3.287592in}{1.408624in}}%
\pgfpathlineto{\pgfqpoint{3.354983in}{1.293267in}}%
\pgfpathlineto{\pgfqpoint{3.422374in}{1.180280in}}%
\pgfpathlineto{\pgfqpoint{3.489765in}{1.071519in}}%
\pgfpathlineto{\pgfqpoint{3.557155in}{0.968769in}}%
\pgfpathlineto{\pgfqpoint{3.624546in}{0.873718in}}%
\pgfpathlineto{\pgfqpoint{3.691937in}{0.787925in}}%
\pgfpathlineto{\pgfqpoint{3.759328in}{0.712801in}}%
\pgfpathlineto{\pgfqpoint{3.826718in}{0.649577in}}%
\pgfpathlineto{\pgfqpoint{3.894109in}{0.599294in}}%
\pgfpathlineto{\pgfqpoint{3.961500in}{0.562775in}}%
\pgfpathlineto{\pgfqpoint{4.028890in}{0.540621in}}%
\pgfpathlineto{\pgfqpoint{4.096281in}{0.533196in}}%
\pgfusepath{stroke}%
\end{pgfscope}%
\begin{pgfscope}%
\pgfsetrectcap%
\pgfsetmiterjoin%
\pgfsetlinewidth{0.803000pt}%
\definecolor{currentstroke}{rgb}{0.000000,0.000000,0.000000}%
\pgfsetstrokecolor{currentstroke}%
\pgfsetdash{}{0pt}%
\pgfpathmoveto{\pgfqpoint{0.629028in}{0.442778in}}%
\pgfpathlineto{\pgfqpoint{0.629028in}{2.431981in}}%
\pgfusepath{stroke}%
\end{pgfscope}%
\begin{pgfscope}%
\pgfsetrectcap%
\pgfsetmiterjoin%
\pgfsetlinewidth{0.803000pt}%
\definecolor{currentstroke}{rgb}{0.000000,0.000000,0.000000}%
\pgfsetstrokecolor{currentstroke}%
\pgfsetdash{}{0pt}%
\pgfpathmoveto{\pgfqpoint{4.261389in}{0.442778in}}%
\pgfpathlineto{\pgfqpoint{4.261389in}{2.431981in}}%
\pgfusepath{stroke}%
\end{pgfscope}%
\begin{pgfscope}%
\pgfsetrectcap%
\pgfsetmiterjoin%
\pgfsetlinewidth{0.803000pt}%
\definecolor{currentstroke}{rgb}{0.000000,0.000000,0.000000}%
\pgfsetstrokecolor{currentstroke}%
\pgfsetdash{}{0pt}%
\pgfpathmoveto{\pgfqpoint{0.629028in}{0.442778in}}%
\pgfpathlineto{\pgfqpoint{4.261389in}{0.442778in}}%
\pgfusepath{stroke}%
\end{pgfscope}%
\begin{pgfscope}%
\pgfsetrectcap%
\pgfsetmiterjoin%
\pgfsetlinewidth{0.803000pt}%
\definecolor{currentstroke}{rgb}{0.000000,0.000000,0.000000}%
\pgfsetstrokecolor{currentstroke}%
\pgfsetdash{}{0pt}%
\pgfpathmoveto{\pgfqpoint{0.629028in}{2.431981in}}%
\pgfpathlineto{\pgfqpoint{4.261389in}{2.431981in}}%
\pgfusepath{stroke}%
\end{pgfscope}%
\begin{pgfscope}%
\pgftext[x=2.445208in,y=2.515314in,,base]{\sffamily\fontsize{12.000000}{14.400000}\selectfont Sine and cosine}%
\end{pgfscope}%
\begin{pgfscope}%
\pgfsetbuttcap%
\pgfsetmiterjoin%
\definecolor{currentfill}{rgb}{1.000000,1.000000,1.000000}%
\pgfsetfillcolor{currentfill}%
\pgfsetfillopacity{0.800000}%
\pgfsetlinewidth{1.003750pt}%
\definecolor{currentstroke}{rgb}{0.800000,0.800000,0.800000}%
\pgfsetstrokecolor{currentstroke}%
\pgfsetstrokeopacity{0.800000}%
\pgfsetdash{}{0pt}%
\pgfpathmoveto{\pgfqpoint{0.726250in}{1.913155in}}%
\pgfpathlineto{\pgfqpoint{1.616455in}{1.913155in}}%
\pgfpathquadraticcurveto{\pgfqpoint{1.644232in}{1.913155in}}{\pgfqpoint{1.644232in}{1.940933in}}%
\pgfpathlineto{\pgfqpoint{1.644232in}{2.334759in}}%
\pgfpathquadraticcurveto{\pgfqpoint{1.644232in}{2.362536in}}{\pgfqpoint{1.616455in}{2.362536in}}%
\pgfpathlineto{\pgfqpoint{0.726250in}{2.362536in}}%
\pgfpathquadraticcurveto{\pgfqpoint{0.698472in}{2.362536in}}{\pgfqpoint{0.698472in}{2.334759in}}%
\pgfpathlineto{\pgfqpoint{0.698472in}{1.940933in}}%
\pgfpathquadraticcurveto{\pgfqpoint{0.698472in}{1.913155in}}{\pgfqpoint{0.726250in}{1.913155in}}%
\pgfpathclose%
\pgfusepath{stroke,fill}%
\end{pgfscope}%
\begin{pgfscope}%
\pgfsetrectcap%
\pgfsetroundjoin%
\pgfsetlinewidth{1.505625pt}%
\definecolor{currentstroke}{rgb}{0.121569,0.466667,0.705882}%
\pgfsetstrokecolor{currentstroke}%
\pgfsetdash{}{0pt}%
\pgfpathmoveto{\pgfqpoint{0.754028in}{2.250069in}}%
\pgfpathlineto{\pgfqpoint{1.031806in}{2.250069in}}%
\pgfusepath{stroke}%
\end{pgfscope}%
\begin{pgfscope}%
\pgftext[x=1.142917in,y=2.201458in,left,base]{\sffamily\fontsize{10.000000}{12.000000}\selectfont sine}%
\end{pgfscope}%
\begin{pgfscope}%
\pgfsetrectcap%
\pgfsetroundjoin%
\pgfsetlinewidth{1.505625pt}%
\definecolor{currentstroke}{rgb}{1.000000,0.498039,0.054902}%
\pgfsetstrokecolor{currentstroke}%
\pgfsetdash{}{0pt}%
\pgfpathmoveto{\pgfqpoint{0.754028in}{2.046212in}}%
\pgfpathlineto{\pgfqpoint{1.031806in}{2.046212in}}%
\pgfusepath{stroke}%
\end{pgfscope}%
\begin{pgfscope}%
\pgftext[x=1.142917in,y=1.997601in,left,base]{\sffamily\fontsize{10.000000}{12.000000}\selectfont cosine}%
\end{pgfscope}%
\end{pgfpicture}%
\makeatother%
\endgroup%

  \caption[Default plot]{The plot, with default parameters before \latexify ing.}
  \label{fig:sincos_no_latex}
\end{figure}

\pyline{figure} is a context manager with the following signature:

\begin{listing}[H]
  \begin{pycode}
    @contextmanager
    def figure(filename, *, directory='img', exts=['pgf', 'png'],
               size=None, mkdir=True):
  \end{pycode}
  \caption[\pyline{lp.figure()} signature]{The signature for \pyline{lp.figure()}.}
  \label{lst:figure}
\end{listing}

Note that the default for \pyline{directory}, \texttt{./img}, is relative to the calling location, and that \pyline{size} is a tuple with the $x$- and $y$-dimensions in inches.

\subsection{With \latexify}
\Cref{fig:sincos_defaults} shows a plot using the default \latexipy\ configuration, with the same typeface as the body, but a size of 8\,pt. To generate it, we can use the code in \Cref{lst:sincos_defaults}.

\begin{listing}[H]
  \begin{pycode*}{highlightlines={1}}
    lp.latexify()

    with figure('sincos'):
        plot_sin_and_cos()
  \end{pycode*}
  \caption[Example of \latexify]{\pyline{lp.latexify()} generates plots that fit well with \LaTeX.}
  \label{lst:sincos_defaults}
\end{listing}

\begin{figure}[H]
  \centering
  %% Creator: Matplotlib, PGF backend
%%
%% To include the figure in your LaTeX document, write
%%   \input{<filename>.pgf}
%%
%% Make sure the required packages are loaded in your preamble
%%   \usepackage{pgf}
%%
%% Figures using additional raster images can only be included by \input if
%% they are in the same directory as the main LaTeX file. For loading figures
%% from other directories you can use the `import` package
%%   \usepackage{import}
%% and then include the figures with
%%   \import{<path to file>}{<filename>.pgf}
%%
%% Matplotlib used the following preamble
%%   \usepackage[utf8x]{inputenc}
%%   \usepackage[T1]{fontenc}
%%   \usepackage{fontspec}
%%
\begingroup%
\makeatletter%
\begin{pgfpicture}%
\pgfpathrectangle{\pgfpointorigin}{\pgfqpoint{4.296389in}{2.655314in}}%
\pgfusepath{use as bounding box, clip}%
\begin{pgfscope}%
\pgfsetbuttcap%
\pgfsetmiterjoin%
\definecolor{currentfill}{rgb}{1.000000,1.000000,1.000000}%
\pgfsetfillcolor{currentfill}%
\pgfsetlinewidth{0.000000pt}%
\definecolor{currentstroke}{rgb}{1.000000,1.000000,1.000000}%
\pgfsetstrokecolor{currentstroke}%
\pgfsetdash{}{0pt}%
\pgfpathmoveto{\pgfqpoint{0.000000in}{0.000000in}}%
\pgfpathlineto{\pgfqpoint{4.296389in}{0.000000in}}%
\pgfpathlineto{\pgfqpoint{4.296389in}{2.655314in}}%
\pgfpathlineto{\pgfqpoint{0.000000in}{2.655314in}}%
\pgfpathclose%
\pgfusepath{fill}%
\end{pgfscope}%
\begin{pgfscope}%
\pgfsetbuttcap%
\pgfsetmiterjoin%
\definecolor{currentfill}{rgb}{1.000000,1.000000,1.000000}%
\pgfsetfillcolor{currentfill}%
\pgfsetlinewidth{0.000000pt}%
\definecolor{currentstroke}{rgb}{0.000000,0.000000,0.000000}%
\pgfsetstrokecolor{currentstroke}%
\pgfsetstrokeopacity{0.000000}%
\pgfsetdash{}{0pt}%
\pgfpathmoveto{\pgfqpoint{0.553146in}{0.350111in}}%
\pgfpathlineto{\pgfqpoint{4.261389in}{0.350111in}}%
\pgfpathlineto{\pgfqpoint{4.261389in}{2.494870in}}%
\pgfpathlineto{\pgfqpoint{0.553146in}{2.494870in}}%
\pgfpathclose%
\pgfusepath{fill}%
\end{pgfscope}%
\begin{pgfscope}%
\pgfsetbuttcap%
\pgfsetroundjoin%
\definecolor{currentfill}{rgb}{0.000000,0.000000,0.000000}%
\pgfsetfillcolor{currentfill}%
\pgfsetlinewidth{0.803000pt}%
\definecolor{currentstroke}{rgb}{0.000000,0.000000,0.000000}%
\pgfsetstrokecolor{currentstroke}%
\pgfsetdash{}{0pt}%
\pgfsys@defobject{currentmarker}{\pgfqpoint{0.000000in}{-0.048611in}}{\pgfqpoint{0.000000in}{0.000000in}}{%
\pgfpathmoveto{\pgfqpoint{0.000000in}{0.000000in}}%
\pgfpathlineto{\pgfqpoint{0.000000in}{-0.048611in}}%
\pgfusepath{stroke,fill}%
}%
\begin{pgfscope}%
\pgfsys@transformshift{0.797672in}{0.350111in}%
\pgfsys@useobject{currentmarker}{}%
\end{pgfscope}%
\end{pgfscope}%
\begin{pgfscope}%
\pgftext[x=0.797672in,y=0.252889in,,top]{\rmfamily\fontsize{8.000000}{9.600000}\selectfont \(\displaystyle -3\)}%
\end{pgfscope}%
\begin{pgfscope}%
\pgfsetbuttcap%
\pgfsetroundjoin%
\definecolor{currentfill}{rgb}{0.000000,0.000000,0.000000}%
\pgfsetfillcolor{currentfill}%
\pgfsetlinewidth{0.803000pt}%
\definecolor{currentstroke}{rgb}{0.000000,0.000000,0.000000}%
\pgfsetstrokecolor{currentstroke}%
\pgfsetdash{}{0pt}%
\pgfsys@defobject{currentmarker}{\pgfqpoint{0.000000in}{-0.048611in}}{\pgfqpoint{0.000000in}{0.000000in}}{%
\pgfpathmoveto{\pgfqpoint{0.000000in}{0.000000in}}%
\pgfpathlineto{\pgfqpoint{0.000000in}{-0.048611in}}%
\pgfusepath{stroke,fill}%
}%
\begin{pgfscope}%
\pgfsys@transformshift{1.334203in}{0.350111in}%
\pgfsys@useobject{currentmarker}{}%
\end{pgfscope}%
\end{pgfscope}%
\begin{pgfscope}%
\pgftext[x=1.334203in,y=0.252889in,,top]{\rmfamily\fontsize{8.000000}{9.600000}\selectfont \(\displaystyle -2\)}%
\end{pgfscope}%
\begin{pgfscope}%
\pgfsetbuttcap%
\pgfsetroundjoin%
\definecolor{currentfill}{rgb}{0.000000,0.000000,0.000000}%
\pgfsetfillcolor{currentfill}%
\pgfsetlinewidth{0.803000pt}%
\definecolor{currentstroke}{rgb}{0.000000,0.000000,0.000000}%
\pgfsetstrokecolor{currentstroke}%
\pgfsetdash{}{0pt}%
\pgfsys@defobject{currentmarker}{\pgfqpoint{0.000000in}{-0.048611in}}{\pgfqpoint{0.000000in}{0.000000in}}{%
\pgfpathmoveto{\pgfqpoint{0.000000in}{0.000000in}}%
\pgfpathlineto{\pgfqpoint{0.000000in}{-0.048611in}}%
\pgfusepath{stroke,fill}%
}%
\begin{pgfscope}%
\pgfsys@transformshift{1.870735in}{0.350111in}%
\pgfsys@useobject{currentmarker}{}%
\end{pgfscope}%
\end{pgfscope}%
\begin{pgfscope}%
\pgftext[x=1.870735in,y=0.252889in,,top]{\rmfamily\fontsize{8.000000}{9.600000}\selectfont \(\displaystyle -1\)}%
\end{pgfscope}%
\begin{pgfscope}%
\pgfsetbuttcap%
\pgfsetroundjoin%
\definecolor{currentfill}{rgb}{0.000000,0.000000,0.000000}%
\pgfsetfillcolor{currentfill}%
\pgfsetlinewidth{0.803000pt}%
\definecolor{currentstroke}{rgb}{0.000000,0.000000,0.000000}%
\pgfsetstrokecolor{currentstroke}%
\pgfsetdash{}{0pt}%
\pgfsys@defobject{currentmarker}{\pgfqpoint{0.000000in}{-0.048611in}}{\pgfqpoint{0.000000in}{0.000000in}}{%
\pgfpathmoveto{\pgfqpoint{0.000000in}{0.000000in}}%
\pgfpathlineto{\pgfqpoint{0.000000in}{-0.048611in}}%
\pgfusepath{stroke,fill}%
}%
\begin{pgfscope}%
\pgfsys@transformshift{2.407267in}{0.350111in}%
\pgfsys@useobject{currentmarker}{}%
\end{pgfscope}%
\end{pgfscope}%
\begin{pgfscope}%
\pgftext[x=2.407267in,y=0.252889in,,top]{\rmfamily\fontsize{8.000000}{9.600000}\selectfont \(\displaystyle 0\)}%
\end{pgfscope}%
\begin{pgfscope}%
\pgfsetbuttcap%
\pgfsetroundjoin%
\definecolor{currentfill}{rgb}{0.000000,0.000000,0.000000}%
\pgfsetfillcolor{currentfill}%
\pgfsetlinewidth{0.803000pt}%
\definecolor{currentstroke}{rgb}{0.000000,0.000000,0.000000}%
\pgfsetstrokecolor{currentstroke}%
\pgfsetdash{}{0pt}%
\pgfsys@defobject{currentmarker}{\pgfqpoint{0.000000in}{-0.048611in}}{\pgfqpoint{0.000000in}{0.000000in}}{%
\pgfpathmoveto{\pgfqpoint{0.000000in}{0.000000in}}%
\pgfpathlineto{\pgfqpoint{0.000000in}{-0.048611in}}%
\pgfusepath{stroke,fill}%
}%
\begin{pgfscope}%
\pgfsys@transformshift{2.943799in}{0.350111in}%
\pgfsys@useobject{currentmarker}{}%
\end{pgfscope}%
\end{pgfscope}%
\begin{pgfscope}%
\pgftext[x=2.943799in,y=0.252889in,,top]{\rmfamily\fontsize{8.000000}{9.600000}\selectfont \(\displaystyle 1\)}%
\end{pgfscope}%
\begin{pgfscope}%
\pgfsetbuttcap%
\pgfsetroundjoin%
\definecolor{currentfill}{rgb}{0.000000,0.000000,0.000000}%
\pgfsetfillcolor{currentfill}%
\pgfsetlinewidth{0.803000pt}%
\definecolor{currentstroke}{rgb}{0.000000,0.000000,0.000000}%
\pgfsetstrokecolor{currentstroke}%
\pgfsetdash{}{0pt}%
\pgfsys@defobject{currentmarker}{\pgfqpoint{0.000000in}{-0.048611in}}{\pgfqpoint{0.000000in}{0.000000in}}{%
\pgfpathmoveto{\pgfqpoint{0.000000in}{0.000000in}}%
\pgfpathlineto{\pgfqpoint{0.000000in}{-0.048611in}}%
\pgfusepath{stroke,fill}%
}%
\begin{pgfscope}%
\pgfsys@transformshift{3.480331in}{0.350111in}%
\pgfsys@useobject{currentmarker}{}%
\end{pgfscope}%
\end{pgfscope}%
\begin{pgfscope}%
\pgftext[x=3.480331in,y=0.252889in,,top]{\rmfamily\fontsize{8.000000}{9.600000}\selectfont \(\displaystyle 2\)}%
\end{pgfscope}%
\begin{pgfscope}%
\pgfsetbuttcap%
\pgfsetroundjoin%
\definecolor{currentfill}{rgb}{0.000000,0.000000,0.000000}%
\pgfsetfillcolor{currentfill}%
\pgfsetlinewidth{0.803000pt}%
\definecolor{currentstroke}{rgb}{0.000000,0.000000,0.000000}%
\pgfsetstrokecolor{currentstroke}%
\pgfsetdash{}{0pt}%
\pgfsys@defobject{currentmarker}{\pgfqpoint{0.000000in}{-0.048611in}}{\pgfqpoint{0.000000in}{0.000000in}}{%
\pgfpathmoveto{\pgfqpoint{0.000000in}{0.000000in}}%
\pgfpathlineto{\pgfqpoint{0.000000in}{-0.048611in}}%
\pgfusepath{stroke,fill}%
}%
\begin{pgfscope}%
\pgfsys@transformshift{4.016863in}{0.350111in}%
\pgfsys@useobject{currentmarker}{}%
\end{pgfscope}%
\end{pgfscope}%
\begin{pgfscope}%
\pgftext[x=4.016863in,y=0.252889in,,top]{\rmfamily\fontsize{8.000000}{9.600000}\selectfont \(\displaystyle 3\)}%
\end{pgfscope}%
\begin{pgfscope}%
\pgftext[x=2.407267in,y=0.098667in,,top]{\rmfamily\fontsize{8.000000}{9.600000}\selectfont \(\displaystyle \theta\)}%
\end{pgfscope}%
\begin{pgfscope}%
\pgfsetbuttcap%
\pgfsetroundjoin%
\definecolor{currentfill}{rgb}{0.000000,0.000000,0.000000}%
\pgfsetfillcolor{currentfill}%
\pgfsetlinewidth{0.803000pt}%
\definecolor{currentstroke}{rgb}{0.000000,0.000000,0.000000}%
\pgfsetstrokecolor{currentstroke}%
\pgfsetdash{}{0pt}%
\pgfsys@defobject{currentmarker}{\pgfqpoint{-0.048611in}{0.000000in}}{\pgfqpoint{0.000000in}{0.000000in}}{%
\pgfpathmoveto{\pgfqpoint{0.000000in}{0.000000in}}%
\pgfpathlineto{\pgfqpoint{-0.048611in}{0.000000in}}%
\pgfusepath{stroke,fill}%
}%
\begin{pgfscope}%
\pgfsys@transformshift{0.553146in}{0.447600in}%
\pgfsys@useobject{currentmarker}{}%
\end{pgfscope}%
\end{pgfscope}%
\begin{pgfscope}%
\pgftext[x=0.154222in,y=0.409044in,left,base]{\rmfamily\fontsize{8.000000}{9.600000}\selectfont \(\displaystyle -1.00\)}%
\end{pgfscope}%
\begin{pgfscope}%
\pgfsetbuttcap%
\pgfsetroundjoin%
\definecolor{currentfill}{rgb}{0.000000,0.000000,0.000000}%
\pgfsetfillcolor{currentfill}%
\pgfsetlinewidth{0.803000pt}%
\definecolor{currentstroke}{rgb}{0.000000,0.000000,0.000000}%
\pgfsetstrokecolor{currentstroke}%
\pgfsetdash{}{0pt}%
\pgfsys@defobject{currentmarker}{\pgfqpoint{-0.048611in}{0.000000in}}{\pgfqpoint{0.000000in}{0.000000in}}{%
\pgfpathmoveto{\pgfqpoint{0.000000in}{0.000000in}}%
\pgfpathlineto{\pgfqpoint{-0.048611in}{0.000000in}}%
\pgfusepath{stroke,fill}%
}%
\begin{pgfscope}%
\pgfsys@transformshift{0.553146in}{0.691385in}%
\pgfsys@useobject{currentmarker}{}%
\end{pgfscope}%
\end{pgfscope}%
\begin{pgfscope}%
\pgftext[x=0.154222in,y=0.652830in,left,base]{\rmfamily\fontsize{8.000000}{9.600000}\selectfont \(\displaystyle -0.75\)}%
\end{pgfscope}%
\begin{pgfscope}%
\pgfsetbuttcap%
\pgfsetroundjoin%
\definecolor{currentfill}{rgb}{0.000000,0.000000,0.000000}%
\pgfsetfillcolor{currentfill}%
\pgfsetlinewidth{0.803000pt}%
\definecolor{currentstroke}{rgb}{0.000000,0.000000,0.000000}%
\pgfsetstrokecolor{currentstroke}%
\pgfsetdash{}{0pt}%
\pgfsys@defobject{currentmarker}{\pgfqpoint{-0.048611in}{0.000000in}}{\pgfqpoint{0.000000in}{0.000000in}}{%
\pgfpathmoveto{\pgfqpoint{0.000000in}{0.000000in}}%
\pgfpathlineto{\pgfqpoint{-0.048611in}{0.000000in}}%
\pgfusepath{stroke,fill}%
}%
\begin{pgfscope}%
\pgfsys@transformshift{0.553146in}{0.935170in}%
\pgfsys@useobject{currentmarker}{}%
\end{pgfscope}%
\end{pgfscope}%
\begin{pgfscope}%
\pgftext[x=0.154222in,y=0.896615in,left,base]{\rmfamily\fontsize{8.000000}{9.600000}\selectfont \(\displaystyle -0.50\)}%
\end{pgfscope}%
\begin{pgfscope}%
\pgfsetbuttcap%
\pgfsetroundjoin%
\definecolor{currentfill}{rgb}{0.000000,0.000000,0.000000}%
\pgfsetfillcolor{currentfill}%
\pgfsetlinewidth{0.803000pt}%
\definecolor{currentstroke}{rgb}{0.000000,0.000000,0.000000}%
\pgfsetstrokecolor{currentstroke}%
\pgfsetdash{}{0pt}%
\pgfsys@defobject{currentmarker}{\pgfqpoint{-0.048611in}{0.000000in}}{\pgfqpoint{0.000000in}{0.000000in}}{%
\pgfpathmoveto{\pgfqpoint{0.000000in}{0.000000in}}%
\pgfpathlineto{\pgfqpoint{-0.048611in}{0.000000in}}%
\pgfusepath{stroke,fill}%
}%
\begin{pgfscope}%
\pgfsys@transformshift{0.553146in}{1.178956in}%
\pgfsys@useobject{currentmarker}{}%
\end{pgfscope}%
\end{pgfscope}%
\begin{pgfscope}%
\pgftext[x=0.154222in,y=1.140400in,left,base]{\rmfamily\fontsize{8.000000}{9.600000}\selectfont \(\displaystyle -0.25\)}%
\end{pgfscope}%
\begin{pgfscope}%
\pgfsetbuttcap%
\pgfsetroundjoin%
\definecolor{currentfill}{rgb}{0.000000,0.000000,0.000000}%
\pgfsetfillcolor{currentfill}%
\pgfsetlinewidth{0.803000pt}%
\definecolor{currentstroke}{rgb}{0.000000,0.000000,0.000000}%
\pgfsetstrokecolor{currentstroke}%
\pgfsetdash{}{0pt}%
\pgfsys@defobject{currentmarker}{\pgfqpoint{-0.048611in}{0.000000in}}{\pgfqpoint{0.000000in}{0.000000in}}{%
\pgfpathmoveto{\pgfqpoint{0.000000in}{0.000000in}}%
\pgfpathlineto{\pgfqpoint{-0.048611in}{0.000000in}}%
\pgfusepath{stroke,fill}%
}%
\begin{pgfscope}%
\pgfsys@transformshift{0.553146in}{1.422741in}%
\pgfsys@useobject{currentmarker}{}%
\end{pgfscope}%
\end{pgfscope}%
\begin{pgfscope}%
\pgftext[x=0.246044in,y=1.384185in,left,base]{\rmfamily\fontsize{8.000000}{9.600000}\selectfont \(\displaystyle 0.00\)}%
\end{pgfscope}%
\begin{pgfscope}%
\pgfsetbuttcap%
\pgfsetroundjoin%
\definecolor{currentfill}{rgb}{0.000000,0.000000,0.000000}%
\pgfsetfillcolor{currentfill}%
\pgfsetlinewidth{0.803000pt}%
\definecolor{currentstroke}{rgb}{0.000000,0.000000,0.000000}%
\pgfsetstrokecolor{currentstroke}%
\pgfsetdash{}{0pt}%
\pgfsys@defobject{currentmarker}{\pgfqpoint{-0.048611in}{0.000000in}}{\pgfqpoint{0.000000in}{0.000000in}}{%
\pgfpathmoveto{\pgfqpoint{0.000000in}{0.000000in}}%
\pgfpathlineto{\pgfqpoint{-0.048611in}{0.000000in}}%
\pgfusepath{stroke,fill}%
}%
\begin{pgfscope}%
\pgfsys@transformshift{0.553146in}{1.666526in}%
\pgfsys@useobject{currentmarker}{}%
\end{pgfscope}%
\end{pgfscope}%
\begin{pgfscope}%
\pgftext[x=0.246044in,y=1.627970in,left,base]{\rmfamily\fontsize{8.000000}{9.600000}\selectfont \(\displaystyle 0.25\)}%
\end{pgfscope}%
\begin{pgfscope}%
\pgfsetbuttcap%
\pgfsetroundjoin%
\definecolor{currentfill}{rgb}{0.000000,0.000000,0.000000}%
\pgfsetfillcolor{currentfill}%
\pgfsetlinewidth{0.803000pt}%
\definecolor{currentstroke}{rgb}{0.000000,0.000000,0.000000}%
\pgfsetstrokecolor{currentstroke}%
\pgfsetdash{}{0pt}%
\pgfsys@defobject{currentmarker}{\pgfqpoint{-0.048611in}{0.000000in}}{\pgfqpoint{0.000000in}{0.000000in}}{%
\pgfpathmoveto{\pgfqpoint{0.000000in}{0.000000in}}%
\pgfpathlineto{\pgfqpoint{-0.048611in}{0.000000in}}%
\pgfusepath{stroke,fill}%
}%
\begin{pgfscope}%
\pgfsys@transformshift{0.553146in}{1.910311in}%
\pgfsys@useobject{currentmarker}{}%
\end{pgfscope}%
\end{pgfscope}%
\begin{pgfscope}%
\pgftext[x=0.246044in,y=1.871756in,left,base]{\rmfamily\fontsize{8.000000}{9.600000}\selectfont \(\displaystyle 0.50\)}%
\end{pgfscope}%
\begin{pgfscope}%
\pgfsetbuttcap%
\pgfsetroundjoin%
\definecolor{currentfill}{rgb}{0.000000,0.000000,0.000000}%
\pgfsetfillcolor{currentfill}%
\pgfsetlinewidth{0.803000pt}%
\definecolor{currentstroke}{rgb}{0.000000,0.000000,0.000000}%
\pgfsetstrokecolor{currentstroke}%
\pgfsetdash{}{0pt}%
\pgfsys@defobject{currentmarker}{\pgfqpoint{-0.048611in}{0.000000in}}{\pgfqpoint{0.000000in}{0.000000in}}{%
\pgfpathmoveto{\pgfqpoint{0.000000in}{0.000000in}}%
\pgfpathlineto{\pgfqpoint{-0.048611in}{0.000000in}}%
\pgfusepath{stroke,fill}%
}%
\begin{pgfscope}%
\pgfsys@transformshift{0.553146in}{2.154096in}%
\pgfsys@useobject{currentmarker}{}%
\end{pgfscope}%
\end{pgfscope}%
\begin{pgfscope}%
\pgftext[x=0.246044in,y=2.115541in,left,base]{\rmfamily\fontsize{8.000000}{9.600000}\selectfont \(\displaystyle 0.75\)}%
\end{pgfscope}%
\begin{pgfscope}%
\pgfsetbuttcap%
\pgfsetroundjoin%
\definecolor{currentfill}{rgb}{0.000000,0.000000,0.000000}%
\pgfsetfillcolor{currentfill}%
\pgfsetlinewidth{0.803000pt}%
\definecolor{currentstroke}{rgb}{0.000000,0.000000,0.000000}%
\pgfsetstrokecolor{currentstroke}%
\pgfsetdash{}{0pt}%
\pgfsys@defobject{currentmarker}{\pgfqpoint{-0.048611in}{0.000000in}}{\pgfqpoint{0.000000in}{0.000000in}}{%
\pgfpathmoveto{\pgfqpoint{0.000000in}{0.000000in}}%
\pgfpathlineto{\pgfqpoint{-0.048611in}{0.000000in}}%
\pgfusepath{stroke,fill}%
}%
\begin{pgfscope}%
\pgfsys@transformshift{0.553146in}{2.397882in}%
\pgfsys@useobject{currentmarker}{}%
\end{pgfscope}%
\end{pgfscope}%
\begin{pgfscope}%
\pgftext[x=0.246044in,y=2.359326in,left,base]{\rmfamily\fontsize{8.000000}{9.600000}\selectfont \(\displaystyle 1.00\)}%
\end{pgfscope}%
\begin{pgfscope}%
\pgftext[x=0.098667in,y=1.422490in,,bottom,rotate=90.000000]{\rmfamily\fontsize{8.000000}{9.600000}\selectfont Value}%
\end{pgfscope}%
\begin{pgfscope}%
\pgfpathrectangle{\pgfqpoint{0.553146in}{0.350111in}}{\pgfqpoint{3.708242in}{2.144759in}} %
\pgfusepath{clip}%
\pgfsetrectcap%
\pgfsetroundjoin%
\pgfsetlinewidth{1.505625pt}%
\definecolor{currentstroke}{rgb}{0.121569,0.466667,0.705882}%
\pgfsetstrokecolor{currentstroke}%
\pgfsetdash{}{0pt}%
\pgfpathmoveto{\pgfqpoint{0.721703in}{1.422741in}}%
\pgfpathlineto{\pgfqpoint{0.790501in}{1.298043in}}%
\pgfpathlineto{\pgfqpoint{0.859300in}{1.175392in}}%
\pgfpathlineto{\pgfqpoint{0.928098in}{1.056803in}}%
\pgfpathlineto{\pgfqpoint{0.996897in}{0.944222in}}%
\pgfpathlineto{\pgfqpoint{1.065695in}{0.839499in}}%
\pgfpathlineto{\pgfqpoint{1.134494in}{0.744352in}}%
\pgfpathlineto{\pgfqpoint{1.203293in}{0.660345in}}%
\pgfpathlineto{\pgfqpoint{1.272091in}{0.588856in}}%
\pgfpathlineto{\pgfqpoint{1.340890in}{0.531060in}}%
\pgfpathlineto{\pgfqpoint{1.409688in}{0.487905in}}%
\pgfpathlineto{\pgfqpoint{1.478487in}{0.460099in}}%
\pgfpathlineto{\pgfqpoint{1.547285in}{0.448101in}}%
\pgfpathlineto{\pgfqpoint{1.616084in}{0.452106in}}%
\pgfpathlineto{\pgfqpoint{1.684882in}{0.472049in}}%
\pgfpathlineto{\pgfqpoint{1.753681in}{0.507602in}}%
\pgfpathlineto{\pgfqpoint{1.822480in}{0.558182in}}%
\pgfpathlineto{\pgfqpoint{1.891278in}{0.622957in}}%
\pgfpathlineto{\pgfqpoint{1.960077in}{0.700865in}}%
\pgfpathlineto{\pgfqpoint{2.028875in}{0.790627in}}%
\pgfpathlineto{\pgfqpoint{2.097674in}{0.890767in}}%
\pgfpathlineto{\pgfqpoint{2.166472in}{0.999643in}}%
\pgfpathlineto{\pgfqpoint{2.235271in}{1.115466in}}%
\pgfpathlineto{\pgfqpoint{2.304069in}{1.236334in}}%
\pgfpathlineto{\pgfqpoint{2.372868in}{1.360263in}}%
\pgfpathlineto{\pgfqpoint{2.441667in}{1.485218in}}%
\pgfpathlineto{\pgfqpoint{2.510465in}{1.609147in}}%
\pgfpathlineto{\pgfqpoint{2.579264in}{1.730016in}}%
\pgfpathlineto{\pgfqpoint{2.648062in}{1.845839in}}%
\pgfpathlineto{\pgfqpoint{2.716861in}{1.954714in}}%
\pgfpathlineto{\pgfqpoint{2.785659in}{2.054855in}}%
\pgfpathlineto{\pgfqpoint{2.854458in}{2.144616in}}%
\pgfpathlineto{\pgfqpoint{2.923257in}{2.222524in}}%
\pgfpathlineto{\pgfqpoint{2.992055in}{2.287300in}}%
\pgfpathlineto{\pgfqpoint{3.060854in}{2.337880in}}%
\pgfpathlineto{\pgfqpoint{3.129652in}{2.373433in}}%
\pgfpathlineto{\pgfqpoint{3.198451in}{2.393376in}}%
\pgfpathlineto{\pgfqpoint{3.267249in}{2.397381in}}%
\pgfpathlineto{\pgfqpoint{3.336048in}{2.385382in}}%
\pgfpathlineto{\pgfqpoint{3.404846in}{2.357577in}}%
\pgfpathlineto{\pgfqpoint{3.473645in}{2.314422in}}%
\pgfpathlineto{\pgfqpoint{3.542444in}{2.256625in}}%
\pgfpathlineto{\pgfqpoint{3.611242in}{2.185137in}}%
\pgfpathlineto{\pgfqpoint{3.680041in}{2.101129in}}%
\pgfpathlineto{\pgfqpoint{3.748839in}{2.005983in}}%
\pgfpathlineto{\pgfqpoint{3.817638in}{1.901260in}}%
\pgfpathlineto{\pgfqpoint{3.886436in}{1.788679in}}%
\pgfpathlineto{\pgfqpoint{3.955235in}{1.670090in}}%
\pgfpathlineto{\pgfqpoint{4.024034in}{1.547439in}}%
\pgfpathlineto{\pgfqpoint{4.092832in}{1.422741in}}%
\pgfusepath{stroke}%
\end{pgfscope}%
\begin{pgfscope}%
\pgfpathrectangle{\pgfqpoint{0.553146in}{0.350111in}}{\pgfqpoint{3.708242in}{2.144759in}} %
\pgfusepath{clip}%
\pgfsetrectcap%
\pgfsetroundjoin%
\pgfsetlinewidth{1.505625pt}%
\definecolor{currentstroke}{rgb}{1.000000,0.498039,0.054902}%
\pgfsetstrokecolor{currentstroke}%
\pgfsetdash{}{0pt}%
\pgfpathmoveto{\pgfqpoint{0.721703in}{0.447600in}}%
\pgfpathlineto{\pgfqpoint{0.790501in}{0.455606in}}%
\pgfpathlineto{\pgfqpoint{0.859300in}{0.479492in}}%
\pgfpathlineto{\pgfqpoint{0.928098in}{0.518866in}}%
\pgfpathlineto{\pgfqpoint{0.996897in}{0.573082in}}%
\pgfpathlineto{\pgfqpoint{1.065695in}{0.641250in}}%
\pgfpathlineto{\pgfqpoint{1.134494in}{0.722249in}}%
\pgfpathlineto{\pgfqpoint{1.203293in}{0.814750in}}%
\pgfpathlineto{\pgfqpoint{1.272091in}{0.917235in}}%
\pgfpathlineto{\pgfqpoint{1.340890in}{1.028020in}}%
\pgfpathlineto{\pgfqpoint{1.409688in}{1.145286in}}%
\pgfpathlineto{\pgfqpoint{1.478487in}{1.267108in}}%
\pgfpathlineto{\pgfqpoint{1.547285in}{1.391486in}}%
\pgfpathlineto{\pgfqpoint{1.616084in}{1.516377in}}%
\pgfpathlineto{\pgfqpoint{1.684882in}{1.639730in}}%
\pgfpathlineto{\pgfqpoint{1.753681in}{1.759520in}}%
\pgfpathlineto{\pgfqpoint{1.822480in}{1.873781in}}%
\pgfpathlineto{\pgfqpoint{1.891278in}{1.980635in}}%
\pgfpathlineto{\pgfqpoint{1.960077in}{2.078329in}}%
\pgfpathlineto{\pgfqpoint{2.028875in}{2.165258in}}%
\pgfpathlineto{\pgfqpoint{2.097674in}{2.239995in}}%
\pgfpathlineto{\pgfqpoint{2.166472in}{2.301312in}}%
\pgfpathlineto{\pgfqpoint{2.235271in}{2.348204in}}%
\pgfpathlineto{\pgfqpoint{2.304069in}{2.379899in}}%
\pgfpathlineto{\pgfqpoint{2.372868in}{2.395878in}}%
\pgfpathlineto{\pgfqpoint{2.441667in}{2.395878in}}%
\pgfpathlineto{\pgfqpoint{2.510465in}{2.379899in}}%
\pgfpathlineto{\pgfqpoint{2.579264in}{2.348204in}}%
\pgfpathlineto{\pgfqpoint{2.648062in}{2.301312in}}%
\pgfpathlineto{\pgfqpoint{2.716861in}{2.239995in}}%
\pgfpathlineto{\pgfqpoint{2.785659in}{2.165258in}}%
\pgfpathlineto{\pgfqpoint{2.854458in}{2.078329in}}%
\pgfpathlineto{\pgfqpoint{2.923257in}{1.980635in}}%
\pgfpathlineto{\pgfqpoint{2.992055in}{1.873781in}}%
\pgfpathlineto{\pgfqpoint{3.060854in}{1.759520in}}%
\pgfpathlineto{\pgfqpoint{3.129652in}{1.639730in}}%
\pgfpathlineto{\pgfqpoint{3.198451in}{1.516377in}}%
\pgfpathlineto{\pgfqpoint{3.267249in}{1.391486in}}%
\pgfpathlineto{\pgfqpoint{3.336048in}{1.267108in}}%
\pgfpathlineto{\pgfqpoint{3.404846in}{1.145286in}}%
\pgfpathlineto{\pgfqpoint{3.473645in}{1.028020in}}%
\pgfpathlineto{\pgfqpoint{3.542444in}{0.917235in}}%
\pgfpathlineto{\pgfqpoint{3.611242in}{0.814750in}}%
\pgfpathlineto{\pgfqpoint{3.680041in}{0.722249in}}%
\pgfpathlineto{\pgfqpoint{3.748839in}{0.641250in}}%
\pgfpathlineto{\pgfqpoint{3.817638in}{0.573082in}}%
\pgfpathlineto{\pgfqpoint{3.886436in}{0.518866in}}%
\pgfpathlineto{\pgfqpoint{3.955235in}{0.479492in}}%
\pgfpathlineto{\pgfqpoint{4.024034in}{0.455606in}}%
\pgfpathlineto{\pgfqpoint{4.092832in}{0.447600in}}%
\pgfusepath{stroke}%
\end{pgfscope}%
\begin{pgfscope}%
\pgfsetrectcap%
\pgfsetmiterjoin%
\pgfsetlinewidth{0.803000pt}%
\definecolor{currentstroke}{rgb}{0.000000,0.000000,0.000000}%
\pgfsetstrokecolor{currentstroke}%
\pgfsetdash{}{0pt}%
\pgfpathmoveto{\pgfqpoint{0.553146in}{0.350111in}}%
\pgfpathlineto{\pgfqpoint{0.553146in}{2.494870in}}%
\pgfusepath{stroke}%
\end{pgfscope}%
\begin{pgfscope}%
\pgfsetrectcap%
\pgfsetmiterjoin%
\pgfsetlinewidth{0.803000pt}%
\definecolor{currentstroke}{rgb}{0.000000,0.000000,0.000000}%
\pgfsetstrokecolor{currentstroke}%
\pgfsetdash{}{0pt}%
\pgfpathmoveto{\pgfqpoint{4.261389in}{0.350111in}}%
\pgfpathlineto{\pgfqpoint{4.261389in}{2.494870in}}%
\pgfusepath{stroke}%
\end{pgfscope}%
\begin{pgfscope}%
\pgfsetrectcap%
\pgfsetmiterjoin%
\pgfsetlinewidth{0.803000pt}%
\definecolor{currentstroke}{rgb}{0.000000,0.000000,0.000000}%
\pgfsetstrokecolor{currentstroke}%
\pgfsetdash{}{0pt}%
\pgfpathmoveto{\pgfqpoint{0.553146in}{0.350111in}}%
\pgfpathlineto{\pgfqpoint{4.261389in}{0.350111in}}%
\pgfusepath{stroke}%
\end{pgfscope}%
\begin{pgfscope}%
\pgfsetrectcap%
\pgfsetmiterjoin%
\pgfsetlinewidth{0.803000pt}%
\definecolor{currentstroke}{rgb}{0.000000,0.000000,0.000000}%
\pgfsetstrokecolor{currentstroke}%
\pgfsetdash{}{0pt}%
\pgfpathmoveto{\pgfqpoint{0.553146in}{2.494870in}}%
\pgfpathlineto{\pgfqpoint{4.261389in}{2.494870in}}%
\pgfusepath{stroke}%
\end{pgfscope}%
\begin{pgfscope}%
\pgftext[x=2.407267in,y=2.578203in,,base]{\rmfamily\fontsize{8.000000}{9.600000}\selectfont Sine and cosine}%
\end{pgfscope}%
\begin{pgfscope}%
\pgfsetbuttcap%
\pgfsetmiterjoin%
\definecolor{currentfill}{rgb}{1.000000,1.000000,1.000000}%
\pgfsetfillcolor{currentfill}%
\pgfsetfillopacity{0.800000}%
\pgfsetlinewidth{1.003750pt}%
\definecolor{currentstroke}{rgb}{0.800000,0.800000,0.800000}%
\pgfsetstrokecolor{currentstroke}%
\pgfsetstrokeopacity{0.800000}%
\pgfsetdash{}{0pt}%
\pgfpathmoveto{\pgfqpoint{0.630924in}{2.096203in}}%
\pgfpathlineto{\pgfqpoint{1.295257in}{2.096203in}}%
\pgfpathquadraticcurveto{\pgfqpoint{1.317479in}{2.096203in}}{\pgfqpoint{1.317479in}{2.118426in}}%
\pgfpathlineto{\pgfqpoint{1.317479in}{2.417092in}}%
\pgfpathquadraticcurveto{\pgfqpoint{1.317479in}{2.439314in}}{\pgfqpoint{1.295257in}{2.439314in}}%
\pgfpathlineto{\pgfqpoint{0.630924in}{2.439314in}}%
\pgfpathquadraticcurveto{\pgfqpoint{0.608702in}{2.439314in}}{\pgfqpoint{0.608702in}{2.417092in}}%
\pgfpathlineto{\pgfqpoint{0.608702in}{2.118426in}}%
\pgfpathquadraticcurveto{\pgfqpoint{0.608702in}{2.096203in}}{\pgfqpoint{0.630924in}{2.096203in}}%
\pgfpathclose%
\pgfusepath{stroke,fill}%
\end{pgfscope}%
\begin{pgfscope}%
\pgfsetrectcap%
\pgfsetroundjoin%
\pgfsetlinewidth{1.505625pt}%
\definecolor{currentstroke}{rgb}{0.121569,0.466667,0.705882}%
\pgfsetstrokecolor{currentstroke}%
\pgfsetdash{}{0pt}%
\pgfpathmoveto{\pgfqpoint{0.653146in}{2.355981in}}%
\pgfpathlineto{\pgfqpoint{0.875368in}{2.355981in}}%
\pgfusepath{stroke}%
\end{pgfscope}%
\begin{pgfscope}%
\pgftext[x=0.964257in,y=2.317092in,left,base]{\rmfamily\fontsize{8.000000}{9.600000}\selectfont sine}%
\end{pgfscope}%
\begin{pgfscope}%
\pgfsetrectcap%
\pgfsetroundjoin%
\pgfsetlinewidth{1.505625pt}%
\definecolor{currentstroke}{rgb}{1.000000,0.498039,0.054902}%
\pgfsetstrokecolor{currentstroke}%
\pgfsetdash{}{0pt}%
\pgfpathmoveto{\pgfqpoint{0.653146in}{2.201092in}}%
\pgfpathlineto{\pgfqpoint{0.875368in}{2.201092in}}%
\pgfusepath{stroke}%
\end{pgfscope}%
\begin{pgfscope}%
\pgftext[x=0.964257in,y=2.162203in,left,base]{\rmfamily\fontsize{8.000000}{9.600000}\selectfont cosine}%
\end{pgfscope}%
\end{pgfpicture}%
\makeatother%
\endgroup%

  \caption[Default \LaTeX\ plot]{The plot, with default parameters after \latexify ing.}
  \label{fig:sincos_defaults}
\end{figure}

\subsection{Custom parameters}
\pyline{lp.latexify()} uses \pyline{lp.PARAMS} by default.
Its values are shown in \Cref{lst:default_params}.

\begin{listing}[H]
  \pyfile[firstline=22, lastline=41]{../latexipy/_latexipy.py}
  \caption[Default \latexify\ settings]{The default parameters changed by \latexify.}
  \label{lst:default_params}
\end{listing}

To change some parameters temporarily, such as the font size, you can use the \pyline{lp.temp_params} context manager. To keep the settings applied, simply pass a different dictionary to \pyline{lp.latexify()}.
Either can be done at any time.
\Cref{lst:new_params_temp} shows an example of increasing the font size temporarily.
\Cref{lst:new_params_permanent} shows an example of increasing the font size permanently.

\begin{listing}[H]
  \pyfile[firstline=87, lastline=90, highlightlines={88}]{examples.py}
  \caption[Using custom \latexify\ settings]{Increasing the font size is as simple as setting it once.}
  \label{lst:new_params_temp}
\end{listing}

\begin{listing}[H]
  \pyfile[firstline=93, lastline=101, highlightlines={94, 95, 98}]{examples.py}
  \caption[Permanently changing \latexify\ settings]{Pass a dictionary to \pyline{lp.latexify()} to permanently change a setting.}
  \label{lst:new_params_permanent}
\end{listing}

Either way, rerunning \Cref{lst:sincos_no_latex} after updating the parameters gives \Cref{fig:sincos_big_font_temp}, with a 10\,pt font.

\begin{figure}[H]
  \centering
  %% Creator: Matplotlib, PGF backend
%%
%% To include the figure in your LaTeX document, write
%%   \input{<filename>.pgf}
%%
%% Make sure the required packages are loaded in your preamble
%%   \usepackage{pgf}
%%
%% Figures using additional raster images can only be included by \input if
%% they are in the same directory as the main LaTeX file. For loading figures
%% from other directories you can use the `import` package
%%   \usepackage{import}
%% and then include the figures with
%%   \import{<path to file>}{<filename>.pgf}
%%
%% Matplotlib used the following preamble
%%   \usepackage[utf8x]{inputenc}
%%   \usepackage[T1]{fontenc}
%%   \usepackage{fontspec}
%%
\begingroup%
\makeatletter%
\begin{pgfpicture}%
\pgfpathrectangle{\pgfpointorigin}{\pgfqpoint{4.296389in}{2.655314in}}%
\pgfusepath{use as bounding box, clip}%
\begin{pgfscope}%
\pgfsetbuttcap%
\pgfsetmiterjoin%
\definecolor{currentfill}{rgb}{1.000000,1.000000,1.000000}%
\pgfsetfillcolor{currentfill}%
\pgfsetlinewidth{0.000000pt}%
\definecolor{currentstroke}{rgb}{1.000000,1.000000,1.000000}%
\pgfsetstrokecolor{currentstroke}%
\pgfsetdash{}{0pt}%
\pgfpathmoveto{\pgfqpoint{0.000000in}{0.000000in}}%
\pgfpathlineto{\pgfqpoint{4.296389in}{0.000000in}}%
\pgfpathlineto{\pgfqpoint{4.296389in}{2.655314in}}%
\pgfpathlineto{\pgfqpoint{0.000000in}{2.655314in}}%
\pgfpathclose%
\pgfusepath{fill}%
\end{pgfscope}%
\begin{pgfscope}%
\pgfsetbuttcap%
\pgfsetmiterjoin%
\definecolor{currentfill}{rgb}{1.000000,1.000000,1.000000}%
\pgfsetfillcolor{currentfill}%
\pgfsetlinewidth{0.000000pt}%
\definecolor{currentstroke}{rgb}{0.000000,0.000000,0.000000}%
\pgfsetstrokecolor{currentstroke}%
\pgfsetstrokeopacity{0.000000}%
\pgfsetdash{}{0pt}%
\pgfpathmoveto{\pgfqpoint{0.561606in}{0.399444in}}%
\pgfpathlineto{\pgfqpoint{4.261389in}{0.399444in}}%
\pgfpathlineto{\pgfqpoint{4.261389in}{2.475592in}}%
\pgfpathlineto{\pgfqpoint{0.561606in}{2.475592in}}%
\pgfpathclose%
\pgfusepath{fill}%
\end{pgfscope}%
\begin{pgfscope}%
\pgfsetbuttcap%
\pgfsetroundjoin%
\definecolor{currentfill}{rgb}{0.000000,0.000000,0.000000}%
\pgfsetfillcolor{currentfill}%
\pgfsetlinewidth{0.803000pt}%
\definecolor{currentstroke}{rgb}{0.000000,0.000000,0.000000}%
\pgfsetstrokecolor{currentstroke}%
\pgfsetdash{}{0pt}%
\pgfsys@defobject{currentmarker}{\pgfqpoint{0.000000in}{-0.048611in}}{\pgfqpoint{0.000000in}{0.000000in}}{%
\pgfpathmoveto{\pgfqpoint{0.000000in}{0.000000in}}%
\pgfpathlineto{\pgfqpoint{0.000000in}{-0.048611in}}%
\pgfusepath{stroke,fill}%
}%
\begin{pgfscope}%
\pgfsys@transformshift{0.805573in}{0.399444in}%
\pgfsys@useobject{currentmarker}{}%
\end{pgfscope}%
\end{pgfscope}%
\begin{pgfscope}%
\pgftext[x=0.805573in,y=0.302222in,,top]{\rmfamily\fontsize{10.000000}{12.000000}\selectfont \(\displaystyle -3\)}%
\end{pgfscope}%
\begin{pgfscope}%
\pgfsetbuttcap%
\pgfsetroundjoin%
\definecolor{currentfill}{rgb}{0.000000,0.000000,0.000000}%
\pgfsetfillcolor{currentfill}%
\pgfsetlinewidth{0.803000pt}%
\definecolor{currentstroke}{rgb}{0.000000,0.000000,0.000000}%
\pgfsetstrokecolor{currentstroke}%
\pgfsetdash{}{0pt}%
\pgfsys@defobject{currentmarker}{\pgfqpoint{0.000000in}{-0.048611in}}{\pgfqpoint{0.000000in}{0.000000in}}{%
\pgfpathmoveto{\pgfqpoint{0.000000in}{0.000000in}}%
\pgfpathlineto{\pgfqpoint{0.000000in}{-0.048611in}}%
\pgfusepath{stroke,fill}%
}%
\begin{pgfscope}%
\pgfsys@transformshift{1.340881in}{0.399444in}%
\pgfsys@useobject{currentmarker}{}%
\end{pgfscope}%
\end{pgfscope}%
\begin{pgfscope}%
\pgftext[x=1.340881in,y=0.302222in,,top]{\rmfamily\fontsize{10.000000}{12.000000}\selectfont \(\displaystyle -2\)}%
\end{pgfscope}%
\begin{pgfscope}%
\pgfsetbuttcap%
\pgfsetroundjoin%
\definecolor{currentfill}{rgb}{0.000000,0.000000,0.000000}%
\pgfsetfillcolor{currentfill}%
\pgfsetlinewidth{0.803000pt}%
\definecolor{currentstroke}{rgb}{0.000000,0.000000,0.000000}%
\pgfsetstrokecolor{currentstroke}%
\pgfsetdash{}{0pt}%
\pgfsys@defobject{currentmarker}{\pgfqpoint{0.000000in}{-0.048611in}}{\pgfqpoint{0.000000in}{0.000000in}}{%
\pgfpathmoveto{\pgfqpoint{0.000000in}{0.000000in}}%
\pgfpathlineto{\pgfqpoint{0.000000in}{-0.048611in}}%
\pgfusepath{stroke,fill}%
}%
\begin{pgfscope}%
\pgfsys@transformshift{1.876189in}{0.399444in}%
\pgfsys@useobject{currentmarker}{}%
\end{pgfscope}%
\end{pgfscope}%
\begin{pgfscope}%
\pgftext[x=1.876189in,y=0.302222in,,top]{\rmfamily\fontsize{10.000000}{12.000000}\selectfont \(\displaystyle -1\)}%
\end{pgfscope}%
\begin{pgfscope}%
\pgfsetbuttcap%
\pgfsetroundjoin%
\definecolor{currentfill}{rgb}{0.000000,0.000000,0.000000}%
\pgfsetfillcolor{currentfill}%
\pgfsetlinewidth{0.803000pt}%
\definecolor{currentstroke}{rgb}{0.000000,0.000000,0.000000}%
\pgfsetstrokecolor{currentstroke}%
\pgfsetdash{}{0pt}%
\pgfsys@defobject{currentmarker}{\pgfqpoint{0.000000in}{-0.048611in}}{\pgfqpoint{0.000000in}{0.000000in}}{%
\pgfpathmoveto{\pgfqpoint{0.000000in}{0.000000in}}%
\pgfpathlineto{\pgfqpoint{0.000000in}{-0.048611in}}%
\pgfusepath{stroke,fill}%
}%
\begin{pgfscope}%
\pgfsys@transformshift{2.411497in}{0.399444in}%
\pgfsys@useobject{currentmarker}{}%
\end{pgfscope}%
\end{pgfscope}%
\begin{pgfscope}%
\pgftext[x=2.411497in,y=0.302222in,,top]{\rmfamily\fontsize{10.000000}{12.000000}\selectfont \(\displaystyle 0\)}%
\end{pgfscope}%
\begin{pgfscope}%
\pgfsetbuttcap%
\pgfsetroundjoin%
\definecolor{currentfill}{rgb}{0.000000,0.000000,0.000000}%
\pgfsetfillcolor{currentfill}%
\pgfsetlinewidth{0.803000pt}%
\definecolor{currentstroke}{rgb}{0.000000,0.000000,0.000000}%
\pgfsetstrokecolor{currentstroke}%
\pgfsetdash{}{0pt}%
\pgfsys@defobject{currentmarker}{\pgfqpoint{0.000000in}{-0.048611in}}{\pgfqpoint{0.000000in}{0.000000in}}{%
\pgfpathmoveto{\pgfqpoint{0.000000in}{0.000000in}}%
\pgfpathlineto{\pgfqpoint{0.000000in}{-0.048611in}}%
\pgfusepath{stroke,fill}%
}%
\begin{pgfscope}%
\pgfsys@transformshift{2.946805in}{0.399444in}%
\pgfsys@useobject{currentmarker}{}%
\end{pgfscope}%
\end{pgfscope}%
\begin{pgfscope}%
\pgftext[x=2.946805in,y=0.302222in,,top]{\rmfamily\fontsize{10.000000}{12.000000}\selectfont \(\displaystyle 1\)}%
\end{pgfscope}%
\begin{pgfscope}%
\pgfsetbuttcap%
\pgfsetroundjoin%
\definecolor{currentfill}{rgb}{0.000000,0.000000,0.000000}%
\pgfsetfillcolor{currentfill}%
\pgfsetlinewidth{0.803000pt}%
\definecolor{currentstroke}{rgb}{0.000000,0.000000,0.000000}%
\pgfsetstrokecolor{currentstroke}%
\pgfsetdash{}{0pt}%
\pgfsys@defobject{currentmarker}{\pgfqpoint{0.000000in}{-0.048611in}}{\pgfqpoint{0.000000in}{0.000000in}}{%
\pgfpathmoveto{\pgfqpoint{0.000000in}{0.000000in}}%
\pgfpathlineto{\pgfqpoint{0.000000in}{-0.048611in}}%
\pgfusepath{stroke,fill}%
}%
\begin{pgfscope}%
\pgfsys@transformshift{3.482113in}{0.399444in}%
\pgfsys@useobject{currentmarker}{}%
\end{pgfscope}%
\end{pgfscope}%
\begin{pgfscope}%
\pgftext[x=3.482113in,y=0.302222in,,top]{\rmfamily\fontsize{10.000000}{12.000000}\selectfont \(\displaystyle 2\)}%
\end{pgfscope}%
\begin{pgfscope}%
\pgfsetbuttcap%
\pgfsetroundjoin%
\definecolor{currentfill}{rgb}{0.000000,0.000000,0.000000}%
\pgfsetfillcolor{currentfill}%
\pgfsetlinewidth{0.803000pt}%
\definecolor{currentstroke}{rgb}{0.000000,0.000000,0.000000}%
\pgfsetstrokecolor{currentstroke}%
\pgfsetdash{}{0pt}%
\pgfsys@defobject{currentmarker}{\pgfqpoint{0.000000in}{-0.048611in}}{\pgfqpoint{0.000000in}{0.000000in}}{%
\pgfpathmoveto{\pgfqpoint{0.000000in}{0.000000in}}%
\pgfpathlineto{\pgfqpoint{0.000000in}{-0.048611in}}%
\pgfusepath{stroke,fill}%
}%
\begin{pgfscope}%
\pgfsys@transformshift{4.017421in}{0.399444in}%
\pgfsys@useobject{currentmarker}{}%
\end{pgfscope}%
\end{pgfscope}%
\begin{pgfscope}%
\pgftext[x=4.017421in,y=0.302222in,,top]{\rmfamily\fontsize{10.000000}{12.000000}\selectfont \(\displaystyle 3\)}%
\end{pgfscope}%
\begin{pgfscope}%
\pgftext[x=2.411497in,y=0.123333in,,top]{\rmfamily\fontsize{10.000000}{12.000000}\selectfont \(\displaystyle \theta\)}%
\end{pgfscope}%
\begin{pgfscope}%
\pgfsetbuttcap%
\pgfsetroundjoin%
\definecolor{currentfill}{rgb}{0.000000,0.000000,0.000000}%
\pgfsetfillcolor{currentfill}%
\pgfsetlinewidth{0.803000pt}%
\definecolor{currentstroke}{rgb}{0.000000,0.000000,0.000000}%
\pgfsetstrokecolor{currentstroke}%
\pgfsetdash{}{0pt}%
\pgfsys@defobject{currentmarker}{\pgfqpoint{-0.048611in}{0.000000in}}{\pgfqpoint{0.000000in}{0.000000in}}{%
\pgfpathmoveto{\pgfqpoint{0.000000in}{0.000000in}}%
\pgfpathlineto{\pgfqpoint{-0.048611in}{0.000000in}}%
\pgfusepath{stroke,fill}%
}%
\begin{pgfscope}%
\pgfsys@transformshift{0.561606in}{0.493815in}%
\pgfsys@useobject{currentmarker}{}%
\end{pgfscope}%
\end{pgfscope}%
\begin{pgfscope}%
\pgftext[x=0.178889in,y=0.445620in,left,base]{\rmfamily\fontsize{10.000000}{12.000000}\selectfont \(\displaystyle -1.0\)}%
\end{pgfscope}%
\begin{pgfscope}%
\pgfsetbuttcap%
\pgfsetroundjoin%
\definecolor{currentfill}{rgb}{0.000000,0.000000,0.000000}%
\pgfsetfillcolor{currentfill}%
\pgfsetlinewidth{0.803000pt}%
\definecolor{currentstroke}{rgb}{0.000000,0.000000,0.000000}%
\pgfsetstrokecolor{currentstroke}%
\pgfsetdash{}{0pt}%
\pgfsys@defobject{currentmarker}{\pgfqpoint{-0.048611in}{0.000000in}}{\pgfqpoint{0.000000in}{0.000000in}}{%
\pgfpathmoveto{\pgfqpoint{0.000000in}{0.000000in}}%
\pgfpathlineto{\pgfqpoint{-0.048611in}{0.000000in}}%
\pgfusepath{stroke,fill}%
}%
\begin{pgfscope}%
\pgfsys@transformshift{0.561606in}{0.965788in}%
\pgfsys@useobject{currentmarker}{}%
\end{pgfscope}%
\end{pgfscope}%
\begin{pgfscope}%
\pgftext[x=0.178889in,y=0.917593in,left,base]{\rmfamily\fontsize{10.000000}{12.000000}\selectfont \(\displaystyle -0.5\)}%
\end{pgfscope}%
\begin{pgfscope}%
\pgfsetbuttcap%
\pgfsetroundjoin%
\definecolor{currentfill}{rgb}{0.000000,0.000000,0.000000}%
\pgfsetfillcolor{currentfill}%
\pgfsetlinewidth{0.803000pt}%
\definecolor{currentstroke}{rgb}{0.000000,0.000000,0.000000}%
\pgfsetstrokecolor{currentstroke}%
\pgfsetdash{}{0pt}%
\pgfsys@defobject{currentmarker}{\pgfqpoint{-0.048611in}{0.000000in}}{\pgfqpoint{0.000000in}{0.000000in}}{%
\pgfpathmoveto{\pgfqpoint{0.000000in}{0.000000in}}%
\pgfpathlineto{\pgfqpoint{-0.048611in}{0.000000in}}%
\pgfusepath{stroke,fill}%
}%
\begin{pgfscope}%
\pgfsys@transformshift{0.561606in}{1.437761in}%
\pgfsys@useobject{currentmarker}{}%
\end{pgfscope}%
\end{pgfscope}%
\begin{pgfscope}%
\pgftext[x=0.286914in,y=1.389566in,left,base]{\rmfamily\fontsize{10.000000}{12.000000}\selectfont \(\displaystyle 0.0\)}%
\end{pgfscope}%
\begin{pgfscope}%
\pgfsetbuttcap%
\pgfsetroundjoin%
\definecolor{currentfill}{rgb}{0.000000,0.000000,0.000000}%
\pgfsetfillcolor{currentfill}%
\pgfsetlinewidth{0.803000pt}%
\definecolor{currentstroke}{rgb}{0.000000,0.000000,0.000000}%
\pgfsetstrokecolor{currentstroke}%
\pgfsetdash{}{0pt}%
\pgfsys@defobject{currentmarker}{\pgfqpoint{-0.048611in}{0.000000in}}{\pgfqpoint{0.000000in}{0.000000in}}{%
\pgfpathmoveto{\pgfqpoint{0.000000in}{0.000000in}}%
\pgfpathlineto{\pgfqpoint{-0.048611in}{0.000000in}}%
\pgfusepath{stroke,fill}%
}%
\begin{pgfscope}%
\pgfsys@transformshift{0.561606in}{1.909734in}%
\pgfsys@useobject{currentmarker}{}%
\end{pgfscope}%
\end{pgfscope}%
\begin{pgfscope}%
\pgftext[x=0.286914in,y=1.861539in,left,base]{\rmfamily\fontsize{10.000000}{12.000000}\selectfont \(\displaystyle 0.5\)}%
\end{pgfscope}%
\begin{pgfscope}%
\pgfsetbuttcap%
\pgfsetroundjoin%
\definecolor{currentfill}{rgb}{0.000000,0.000000,0.000000}%
\pgfsetfillcolor{currentfill}%
\pgfsetlinewidth{0.803000pt}%
\definecolor{currentstroke}{rgb}{0.000000,0.000000,0.000000}%
\pgfsetstrokecolor{currentstroke}%
\pgfsetdash{}{0pt}%
\pgfsys@defobject{currentmarker}{\pgfqpoint{-0.048611in}{0.000000in}}{\pgfqpoint{0.000000in}{0.000000in}}{%
\pgfpathmoveto{\pgfqpoint{0.000000in}{0.000000in}}%
\pgfpathlineto{\pgfqpoint{-0.048611in}{0.000000in}}%
\pgfusepath{stroke,fill}%
}%
\begin{pgfscope}%
\pgfsys@transformshift{0.561606in}{2.381707in}%
\pgfsys@useobject{currentmarker}{}%
\end{pgfscope}%
\end{pgfscope}%
\begin{pgfscope}%
\pgftext[x=0.286914in,y=2.333512in,left,base]{\rmfamily\fontsize{10.000000}{12.000000}\selectfont \(\displaystyle 1.0\)}%
\end{pgfscope}%
\begin{pgfscope}%
\pgftext[x=0.123333in,y=1.437518in,,bottom,rotate=90.000000]{\rmfamily\fontsize{10.000000}{12.000000}\selectfont Value}%
\end{pgfscope}%
\begin{pgfscope}%
\pgfpathrectangle{\pgfqpoint{0.561606in}{0.399444in}}{\pgfqpoint{3.699783in}{2.076148in}} %
\pgfusepath{clip}%
\pgfsetrectcap%
\pgfsetroundjoin%
\pgfsetlinewidth{1.505625pt}%
\definecolor{currentstroke}{rgb}{0.121569,0.466667,0.705882}%
\pgfsetstrokecolor{currentstroke}%
\pgfsetdash{}{0pt}%
\pgfpathmoveto{\pgfqpoint{0.729778in}{1.437761in}}%
\pgfpathlineto{\pgfqpoint{0.798419in}{1.317051in}}%
\pgfpathlineto{\pgfqpoint{0.867061in}{1.198324in}}%
\pgfpathlineto{\pgfqpoint{0.935702in}{1.083529in}}%
\pgfpathlineto{\pgfqpoint{1.004344in}{0.974550in}}%
\pgfpathlineto{\pgfqpoint{1.072986in}{0.873176in}}%
\pgfpathlineto{\pgfqpoint{1.141627in}{0.781074in}}%
\pgfpathlineto{\pgfqpoint{1.210269in}{0.699754in}}%
\pgfpathlineto{\pgfqpoint{1.278911in}{0.630552in}}%
\pgfpathlineto{\pgfqpoint{1.347552in}{0.574604in}}%
\pgfpathlineto{\pgfqpoint{1.416194in}{0.532830in}}%
\pgfpathlineto{\pgfqpoint{1.484835in}{0.505914in}}%
\pgfpathlineto{\pgfqpoint{1.553477in}{0.494300in}}%
\pgfpathlineto{\pgfqpoint{1.622119in}{0.498176in}}%
\pgfpathlineto{\pgfqpoint{1.690760in}{0.517481in}}%
\pgfpathlineto{\pgfqpoint{1.759402in}{0.551897in}}%
\pgfpathlineto{\pgfqpoint{1.828043in}{0.600859in}}%
\pgfpathlineto{\pgfqpoint{1.896685in}{0.663562in}}%
\pgfpathlineto{\pgfqpoint{1.965327in}{0.738978in}}%
\pgfpathlineto{\pgfqpoint{2.033968in}{0.825868in}}%
\pgfpathlineto{\pgfqpoint{2.102610in}{0.922805in}}%
\pgfpathlineto{\pgfqpoint{2.171251in}{1.028198in}}%
\pgfpathlineto{\pgfqpoint{2.239893in}{1.140315in}}%
\pgfpathlineto{\pgfqpoint{2.308535in}{1.257317in}}%
\pgfpathlineto{\pgfqpoint{2.377176in}{1.377282in}}%
\pgfpathlineto{\pgfqpoint{2.445818in}{1.498239in}}%
\pgfpathlineto{\pgfqpoint{2.514460in}{1.618204in}}%
\pgfpathlineto{\pgfqpoint{2.583101in}{1.735206in}}%
\pgfpathlineto{\pgfqpoint{2.651743in}{1.847323in}}%
\pgfpathlineto{\pgfqpoint{2.720384in}{1.952716in}}%
\pgfpathlineto{\pgfqpoint{2.789026in}{2.049653in}}%
\pgfpathlineto{\pgfqpoint{2.857668in}{2.136543in}}%
\pgfpathlineto{\pgfqpoint{2.926309in}{2.211959in}}%
\pgfpathlineto{\pgfqpoint{2.994951in}{2.274662in}}%
\pgfpathlineto{\pgfqpoint{3.063592in}{2.323624in}}%
\pgfpathlineto{\pgfqpoint{3.132234in}{2.358040in}}%
\pgfpathlineto{\pgfqpoint{3.200876in}{2.377345in}}%
\pgfpathlineto{\pgfqpoint{3.269517in}{2.381222in}}%
\pgfpathlineto{\pgfqpoint{3.338159in}{2.369607in}}%
\pgfpathlineto{\pgfqpoint{3.406800in}{2.342691in}}%
\pgfpathlineto{\pgfqpoint{3.475442in}{2.300917in}}%
\pgfpathlineto{\pgfqpoint{3.544084in}{2.244969in}}%
\pgfpathlineto{\pgfqpoint{3.612725in}{2.175767in}}%
\pgfpathlineto{\pgfqpoint{3.681367in}{2.094447in}}%
\pgfpathlineto{\pgfqpoint{3.750009in}{2.002345in}}%
\pgfpathlineto{\pgfqpoint{3.818650in}{1.900971in}}%
\pgfpathlineto{\pgfqpoint{3.887292in}{1.791992in}}%
\pgfpathlineto{\pgfqpoint{3.955933in}{1.677197in}}%
\pgfpathlineto{\pgfqpoint{4.024575in}{1.558470in}}%
\pgfpathlineto{\pgfqpoint{4.093217in}{1.437761in}}%
\pgfusepath{stroke}%
\end{pgfscope}%
\begin{pgfscope}%
\pgfpathrectangle{\pgfqpoint{0.561606in}{0.399444in}}{\pgfqpoint{3.699783in}{2.076148in}} %
\pgfusepath{clip}%
\pgfsetrectcap%
\pgfsetroundjoin%
\pgfsetlinewidth{1.505625pt}%
\definecolor{currentstroke}{rgb}{1.000000,0.498039,0.054902}%
\pgfsetstrokecolor{currentstroke}%
\pgfsetdash{}{0pt}%
\pgfpathmoveto{\pgfqpoint{0.729778in}{0.493815in}}%
\pgfpathlineto{\pgfqpoint{0.798419in}{0.501564in}}%
\pgfpathlineto{\pgfqpoint{0.867061in}{0.524686in}}%
\pgfpathlineto{\pgfqpoint{0.935702in}{0.562801in}}%
\pgfpathlineto{\pgfqpoint{1.004344in}{0.615283in}}%
\pgfpathlineto{\pgfqpoint{1.072986in}{0.681269in}}%
\pgfpathlineto{\pgfqpoint{1.141627in}{0.759678in}}%
\pgfpathlineto{\pgfqpoint{1.210269in}{0.849220in}}%
\pgfpathlineto{\pgfqpoint{1.278911in}{0.948426in}}%
\pgfpathlineto{\pgfqpoint{1.347552in}{1.055667in}}%
\pgfpathlineto{\pgfqpoint{1.416194in}{1.169182in}}%
\pgfpathlineto{\pgfqpoint{1.484835in}{1.287107in}}%
\pgfpathlineto{\pgfqpoint{1.553477in}{1.407506in}}%
\pgfpathlineto{\pgfqpoint{1.622119in}{1.528401in}}%
\pgfpathlineto{\pgfqpoint{1.690760in}{1.647808in}}%
\pgfpathlineto{\pgfqpoint{1.759402in}{1.763767in}}%
\pgfpathlineto{\pgfqpoint{1.828043in}{1.874372in}}%
\pgfpathlineto{\pgfqpoint{1.896685in}{1.977808in}}%
\pgfpathlineto{\pgfqpoint{1.965327in}{2.072376in}}%
\pgfpathlineto{\pgfqpoint{2.033968in}{2.156524in}}%
\pgfpathlineto{\pgfqpoint{2.102610in}{2.228870in}}%
\pgfpathlineto{\pgfqpoint{2.171251in}{2.288227in}}%
\pgfpathlineto{\pgfqpoint{2.239893in}{2.333618in}}%
\pgfpathlineto{\pgfqpoint{2.308535in}{2.364299in}}%
\pgfpathlineto{\pgfqpoint{2.377176in}{2.379767in}}%
\pgfpathlineto{\pgfqpoint{2.445818in}{2.379767in}}%
\pgfpathlineto{\pgfqpoint{2.514460in}{2.364299in}}%
\pgfpathlineto{\pgfqpoint{2.583101in}{2.333618in}}%
\pgfpathlineto{\pgfqpoint{2.651743in}{2.288227in}}%
\pgfpathlineto{\pgfqpoint{2.720384in}{2.228870in}}%
\pgfpathlineto{\pgfqpoint{2.789026in}{2.156524in}}%
\pgfpathlineto{\pgfqpoint{2.857668in}{2.072376in}}%
\pgfpathlineto{\pgfqpoint{2.926309in}{1.977808in}}%
\pgfpathlineto{\pgfqpoint{2.994951in}{1.874372in}}%
\pgfpathlineto{\pgfqpoint{3.063592in}{1.763767in}}%
\pgfpathlineto{\pgfqpoint{3.132234in}{1.647808in}}%
\pgfpathlineto{\pgfqpoint{3.200876in}{1.528401in}}%
\pgfpathlineto{\pgfqpoint{3.269517in}{1.407506in}}%
\pgfpathlineto{\pgfqpoint{3.338159in}{1.287107in}}%
\pgfpathlineto{\pgfqpoint{3.406800in}{1.169182in}}%
\pgfpathlineto{\pgfqpoint{3.475442in}{1.055667in}}%
\pgfpathlineto{\pgfqpoint{3.544084in}{0.948426in}}%
\pgfpathlineto{\pgfqpoint{3.612725in}{0.849220in}}%
\pgfpathlineto{\pgfqpoint{3.681367in}{0.759678in}}%
\pgfpathlineto{\pgfqpoint{3.750009in}{0.681269in}}%
\pgfpathlineto{\pgfqpoint{3.818650in}{0.615283in}}%
\pgfpathlineto{\pgfqpoint{3.887292in}{0.562801in}}%
\pgfpathlineto{\pgfqpoint{3.955933in}{0.524686in}}%
\pgfpathlineto{\pgfqpoint{4.024575in}{0.501564in}}%
\pgfpathlineto{\pgfqpoint{4.093217in}{0.493815in}}%
\pgfusepath{stroke}%
\end{pgfscope}%
\begin{pgfscope}%
\pgfsetrectcap%
\pgfsetmiterjoin%
\pgfsetlinewidth{0.803000pt}%
\definecolor{currentstroke}{rgb}{0.000000,0.000000,0.000000}%
\pgfsetstrokecolor{currentstroke}%
\pgfsetdash{}{0pt}%
\pgfpathmoveto{\pgfqpoint{0.561606in}{0.399444in}}%
\pgfpathlineto{\pgfqpoint{0.561606in}{2.475592in}}%
\pgfusepath{stroke}%
\end{pgfscope}%
\begin{pgfscope}%
\pgfsetrectcap%
\pgfsetmiterjoin%
\pgfsetlinewidth{0.803000pt}%
\definecolor{currentstroke}{rgb}{0.000000,0.000000,0.000000}%
\pgfsetstrokecolor{currentstroke}%
\pgfsetdash{}{0pt}%
\pgfpathmoveto{\pgfqpoint{4.261389in}{0.399444in}}%
\pgfpathlineto{\pgfqpoint{4.261389in}{2.475592in}}%
\pgfusepath{stroke}%
\end{pgfscope}%
\begin{pgfscope}%
\pgfsetrectcap%
\pgfsetmiterjoin%
\pgfsetlinewidth{0.803000pt}%
\definecolor{currentstroke}{rgb}{0.000000,0.000000,0.000000}%
\pgfsetstrokecolor{currentstroke}%
\pgfsetdash{}{0pt}%
\pgfpathmoveto{\pgfqpoint{0.561606in}{0.399444in}}%
\pgfpathlineto{\pgfqpoint{4.261389in}{0.399444in}}%
\pgfusepath{stroke}%
\end{pgfscope}%
\begin{pgfscope}%
\pgfsetrectcap%
\pgfsetmiterjoin%
\pgfsetlinewidth{0.803000pt}%
\definecolor{currentstroke}{rgb}{0.000000,0.000000,0.000000}%
\pgfsetstrokecolor{currentstroke}%
\pgfsetdash{}{0pt}%
\pgfpathmoveto{\pgfqpoint{0.561606in}{2.475592in}}%
\pgfpathlineto{\pgfqpoint{4.261389in}{2.475592in}}%
\pgfusepath{stroke}%
\end{pgfscope}%
\begin{pgfscope}%
\pgftext[x=2.411497in,y=2.558925in,,base]{\rmfamily\fontsize{10.000000}{12.000000}\selectfont Sine and cosine}%
\end{pgfscope}%
\begin{pgfscope}%
\pgfsetbuttcap%
\pgfsetmiterjoin%
\definecolor{currentfill}{rgb}{1.000000,1.000000,1.000000}%
\pgfsetfillcolor{currentfill}%
\pgfsetfillopacity{0.800000}%
\pgfsetlinewidth{1.003750pt}%
\definecolor{currentstroke}{rgb}{0.800000,0.800000,0.800000}%
\pgfsetstrokecolor{currentstroke}%
\pgfsetstrokeopacity{0.800000}%
\pgfsetdash{}{0pt}%
\pgfpathmoveto{\pgfqpoint{0.658828in}{1.977259in}}%
\pgfpathlineto{\pgfqpoint{1.466606in}{1.977259in}}%
\pgfpathquadraticcurveto{\pgfqpoint{1.494383in}{1.977259in}}{\pgfqpoint{1.494383in}{2.005037in}}%
\pgfpathlineto{\pgfqpoint{1.494383in}{2.378370in}}%
\pgfpathquadraticcurveto{\pgfqpoint{1.494383in}{2.406147in}}{\pgfqpoint{1.466606in}{2.406147in}}%
\pgfpathlineto{\pgfqpoint{0.658828in}{2.406147in}}%
\pgfpathquadraticcurveto{\pgfqpoint{0.631050in}{2.406147in}}{\pgfqpoint{0.631050in}{2.378370in}}%
\pgfpathlineto{\pgfqpoint{0.631050in}{2.005037in}}%
\pgfpathquadraticcurveto{\pgfqpoint{0.631050in}{1.977259in}}{\pgfqpoint{0.658828in}{1.977259in}}%
\pgfpathclose%
\pgfusepath{stroke,fill}%
\end{pgfscope}%
\begin{pgfscope}%
\pgfsetrectcap%
\pgfsetroundjoin%
\pgfsetlinewidth{1.505625pt}%
\definecolor{currentstroke}{rgb}{0.121569,0.466667,0.705882}%
\pgfsetstrokecolor{currentstroke}%
\pgfsetdash{}{0pt}%
\pgfpathmoveto{\pgfqpoint{0.686606in}{2.301981in}}%
\pgfpathlineto{\pgfqpoint{0.964383in}{2.301981in}}%
\pgfusepath{stroke}%
\end{pgfscope}%
\begin{pgfscope}%
\pgftext[x=1.075495in,y=2.253370in,left,base]{\rmfamily\fontsize{10.000000}{12.000000}\selectfont sine}%
\end{pgfscope}%
\begin{pgfscope}%
\pgfsetrectcap%
\pgfsetroundjoin%
\pgfsetlinewidth{1.505625pt}%
\definecolor{currentstroke}{rgb}{1.000000,0.498039,0.054902}%
\pgfsetstrokecolor{currentstroke}%
\pgfsetdash{}{0pt}%
\pgfpathmoveto{\pgfqpoint{0.686606in}{2.108370in}}%
\pgfpathlineto{\pgfqpoint{0.964383in}{2.108370in}}%
\pgfusepath{stroke}%
\end{pgfscope}%
\begin{pgfscope}%
\pgftext[x=1.075495in,y=2.059759in,left,base]{\rmfamily\fontsize{10.000000}{12.000000}\selectfont cosine}%
\end{pgfscope}%
\end{pgfpicture}%
\makeatother%
\endgroup%

  \caption[Custom \LaTeX\ plot]{The plot, with the font size increased.}
  \label{fig:sincos_big_font_temp}
\end{figure}

\subsection{Reverting the settings}

To revert the settings, just run \pyline{lp.revert()}.

\section{Programming tips}
\label{sec:partial}
If you keep passing the same arguments to \pyline{lp.figure()} (for example, an output directory, a set of filetypes, or a certain size), you can save it for reuse by using the \pyline{partial} function from the \pyline{functools} module, as shown in \Cref{lst:partial}.
After that, you can use it just like \pyline{lp.figure()}.
Note that you would not be able to redifine an argument that you had previously applied.

Here, \pyline{DIRECTORY} refers to any output directory, either as a string, or as a \pyline{pathlib.Path()} object.

\begin{listing}[H]
  \begin{pycode*}{highlightlines={1, 3}}
    figure = partial(lp.figure, directory=DIRECTORY)

    with figure('sincos_partial'):
        plot_sin_and_cos()
  \end{pycode*}
  \caption[Using partial function application]{The \pyline{sincos_partial} plots will be stored in \pyline{DIRECTORY}.}
  \label{lst:partial}
\end{listing}

\section{Using in \LaTeX}
To include a PGF file in your \LaTeX document, make sure that the \mintinline{latex}{pgf} package is loaded in the preamble:

\mint{latex}|\usepackage{pgf}|

After that, you can include it in the correct location with:

\mint{latex}|\input{<filename>.pgf}|

\Cref{lst:latex} shows a minimum working example of adding an image within a figure.

\begin{listing}[H]
  \begin{minted}[highlightlines={3, 8}]{latex}
    \documentclass{article}

    \usepackage{pgf}

    \begin{document}
      \begin{figure}[h]
        \centering
        \input{img/filename.pgf}
        \caption[LOF caption]{Regular caption.}
        \label{fig:pgf_example}
      \end{figure}
    \end{document}
  \end{minted}
  \caption[\LaTeX\ Minimum Working Example]{A minimum working example of using PGF with \LaTeX.}
  \label{lst:latex}
\end{listing}

Note that figures using additional raster images can only be included by \mintinline{latex}{\input} if they are in the same directory as the main \LaTeX file.
To load figures from other directories, you can use the \mintinline{latex}{import} package instead.

\begin{minted}{latex}
  \usepackage{import}
  \import{<path to file>}{<filename>.pgf}
\end{minted}

An example is given in \Cref{lst:import}.

\begin{listing}[H]
  \begin{minted}[highlightlines={3, 4, 9}]{latex}
    \documentclass{article}

    \usepackage{import}
    \usepackage{pgf}

    \begin{document}
      \begin{figure}[h]
        \centering
        \import{/path/to/file/}{filename.pgf}  % Note trailing slash.
        \caption[LOF caption]{Regular caption.}
        \label{fig:pgf_example}
      \end{figure}
    \end{document}
  \end{minted}
  \caption[Importing PGF with raster]{Importing a raster-using PGF from a different directory requires the \mintinline{latex}{\import} package.}
  \label{lst:import}
\end{listing}


\end{document}
